\chapter{埃瓦里斯特 \textbullet 伽罗瓦 和他的时代}



\section{1811—1830年}
\begin{flushright}
	“他被数学的鬼魅迷住了心窍……”
	
	伽罗瓦的教师之一
\end{flushright}

离巴黎十八公里 , 有一座布尔-拉-林(Bourg-la-Reine)小城 ,  现在还是象十九世纪初叶那样十分宁静 . 大街两旁至今还峙立着 几幢从令人难忘的往昔完整地幸存下来的、门上有宽檐的尖顶房屋;城里依然是那几条用伊尔-德-法兰西(l'le-de-France)|\footnote{伊尔-德-法兰西(法学西岛)一十八世纪末资产阶级革命前的法国一省 , 法历史的中心 , 该省的区域现在包括有塞纳-瓦斯省和巴黎市等行政单位 . ——译者}的玫 瑰色岩石砌成的马路 . 那家旅馆仍然挂着“穿靴猫旅社”的招牌 ,  还有那有方院子的带列柱廊的教堂 . 那市政府与1829年相比似乎有点逊色 , 但实际上 , 自从墙上钉上了一块题着“伽罗瓦先生 , 本市十五年常任市长——市民敬立”等字样的纪念牌以来 , 它的外表几乎毫无改变 . 布尔-拉-林城里有一条伽罗瓦街 , 是纪念同一个人——数学家的父亲尼古拉-加布里埃尔 \textbullet 伽罗瓦(Nicalas-Ga-briel Galois)的 . 

大街第54号房的正面另有一块纪念牌 , 写着《法国著名数学 家埃瓦里斯特 \textbullet 伽罗瓦 , 生于此 . 卒年二十岁 , 1811—1832年气这就是埃瓦里斯特 \textbullet 伽罗瓦出生的房子 . 纪念牌是在1909年6 月13日设置的 . 我们对于这个表示敬意的纪念物应当感激一位 布尔-拉-林的居民的急公好义 , 他是当时巴黎大学数学系教授 . 出席纪念牌揭幕仪式的有两位数学家 , 朱利-汤内里(Jules Tannery) 和科学院常任秘书加斯通 \textbullet 达尔布(Gaston Darboux) . 他 们两人都在埃瓦里斯特被开除出校的师范大学里念过书 . 

伽罗瓦的家庭成员 , 除伽罗瓦本人外 , 都埋葬在布尔-拉-林的 墓地里 . 埃瓦里斯特-伽罗瓦安葬在蒙帕尔纳斯(Montparnasse) 墓园的公墓里 . 

尼古拉-加布里埃尔 \textbullet 伽罗瓦在布尔-拉-林主持过供少年就 学的学校 . 这所学校早在旧制度\footnote{即指1789年法国资产阶级革命之前 . ——译者}时期就已创办 , 从那时候起 , 总 是一成不变地由伽罗瓦家族中某一个成员担任该校校长 . 革命 后 , 布尔-拉-林改称为布尔-厄加利特(Bourg-I'Egalite)\footnote{布尔-拉-林(Bourg-la-Reine)意为皇城;布尔-厄加利(Bourg-l'Égalité)意为 平等城 . ——俄译者} , 老伽罗 瓦的学校也改为巴黎学区的中学之一;但尼古拉-加布里埃尔 \textbullet 伽罗瓦仍旧当校长 . 在“百日”\footnote{“百日”——拿破仑一世从厄尔巴岛逃回后 , 第二次统治法国达百日(1815年 3月14日至6月22日) , 在滑扶卢失败后 , 被迫再次退位 . 在“百日”王朝以后 , 拿破仑 被流放到圣海伦岛 . ——译者}王朝期间 , 市民选他当该市市 长 . 老伽罗瓦如此普遍地受人欢迎 , 甚至内务部长也不能不注意 到这种情况:即使在复辟\footnote{复辟——法国历史上波旁王朝被资产阶级革命推翻后 , 在外国武装干涉下恢 复帝位的时期 . 复辟有两次 , 第一次是在1814年4月至1815年3月 , 第二次是“百 日”以后开始 , 从1815年7月至1830年7月 . 复辟制度由于1830年的资产阶级的7 月革命而被消灭 . ——译者}时期老伽罗瓦仍然保持市长的职务 . 

尼古拉-加布里埃尔-伽罗瓦属于自由党人 , 在当时 , 这首先 意味着他对恢复旧制度的不满 . 在旧制度下政权属于专制君主 , 而君主则被认为是上帝在人间的全权代理人 . 在当时 , 拿破仑主 义者都被看作是自由党人 , 因为他们是争取君主立宪的第一批斗士 . 他们的理想体现在可以从反面理解的“立宪”一词中 . 至于具 体行动 , 他们则支持大资产阶级 , 也就是从法国大革命时期以来 , 掌握有实际权力的资产阶级 . 实际上 , 大资产阶级上层分子曾起 过幕后政府的作用 , 同时他们实力如此强大 , 甚至在外交政策方针 上也可以觉察到他们的影响 , 譬如 , 他们经常企图在欧洲各国首都 制造有利的社会舆论 . 在复辟时期 , 从立宪的拥护者的自由党人 联盟中分裂出一个小派别 , 尽管人数不多 , 但它是由优秀分子组成 的 . 这批少数人组成了共和党 , 后来 , 埃瓦里斯特 \textbullet 伽罗瓦就属于 这个党 . 

离第54号房几公尺地方 , 大街的斜对面 , 有一座属于德芒特 (Demante)家的住宅 . 尼古拉-加布里埃尔娶法官托马斯-加布 里埃尔 \textbullet 德芒特(Thomas-Gabriel Demante)的女儿玛利亚-阿代 累达(Marie-Adélaide)为妻 . 这个家庭出了几个杰出的法律系教 授;德芒特家庭的一个成员在1848年以后当过国民议会的议员 ,  但他们之中没有任何人对伽罗瓦的遭遇表示过任何关怀 . 

埃瓦里斯特 \textbullet 伽罗瓦生于1811年10月26日 . 据说 , 玛利 亚-阿代累达 \textbullet 伽罗瓦曾积极参预儿子的教育 . 作为古代文化的 热烈爱好者 , 她把从拉丁和希腊文学中汲取来的英勇典范介绍给 她的儿子 . 迄今保存的唯一的书面文件证实了这种报导 . 1848年 发表在《皮托雷斯克画报》(Magasin Pittoresque)上有关伽罗瓦的 传记中 , 特别谈到“在他的生活中有伟人传记中所习见的相同情 况:伽罗瓦的第一位教师是他的母亲 , 一个聪明兼有好教养的妇女 , 当他还在童稚时 , 她一直给他上课 . ”然而在埃瓦里斯特-伽罗瓦的书信里却没有提到他的母亲 . 同时 , 腊斯拜\footnote{F . Raspail , 1794—1878 , 法国著名政治家、化学家和医生 , 积极参加1830年7 月革命和帝制时期的共和运动 . ——译者}——他和伽罗瓦的关系始末迄今还不清楚——说他和伽罗瓦是圣佩拉吉(Sainte-Pélaqie)监狱中同监的难友 , 伽罗瓦向他说过:“父亲是他的一切” . 

1823年10月 , 伽罗瓦年满十二岁时 , 离开了双亲 , 考入路易-勒-格兰皇家中学(现改为路易-勒-格兰高等专科学校) . 就在这 里 , 他夹在新同学中间在学校里上了生活的第一课 . 在这所中学 学习的年轻人都是出身于资产阶级上层圈子的家庭的 . 他们的父 亲——银行家、工业家、达官显贵——经常左右着自由党人的政 策 . 这些集团的权力伸展得非常之远 . 可是 , 他们仍然不满足于 凭借自己的地位所获得的利润 , 他们千方百计地设法巩固自己的 特权 . 这班人和出身于平民的人(他们称之为“流氓”)一样 , 对贵 族阶级感到深恶痛绝 . 在大学里 , 在中小学里 , 有时干脆就在街 上 , 学生们谈起自由党人所认为是“革命”的话题 . 这种酝酿对资 产阶级有利 , 因为它使资产阶级的敌人不断地感受到威胁 . 中学 的学生都以老同学为榜样 . 可以想象得到 , 埃瓦里斯特-伽罗瓦 在他们当中会感到自己是很孤独的 . 

如果说 , 我们对伽罗瓦的童年几乎一无所知——从他的家庭 成员那里只知道他“有才能、认真、热心” , 那么他的老师们却保存 了很多有关他在中学度过头几年生活的回忆录和笔记 . 如果这些 笔记只证明他们对埃瓦里斯特-伽罗瓦的不友好的态度 , 那就尽 可不重视这些笔记 . 但情况并非如此 . 伽罗瓦的老师们发现他们 的这位学生有“杰出的才干” , 同时又认为他“举止不凡” , 他“为人 乖僻、古怪、过分多嘴” . 有人把这种性格看作青年期神经过敏的 标志 . 我们则认为(伽罗瓦的一生经历就是明显的证据) , 这个孩 子有个性 , 而且当时早已显露出求知的精神 . 

伽罗瓦在路易-勒-格兰皇家中学领奖学金 , 完全靠公费生活 .  在第四、第三和第二年级\footnote{法国中学的年级编号与我国学校所惯用的年级编号正相反 , 即一年级是最高 年级 , 而不是最低年级 . ——译者}时 , 他是优等生 , 在希腊语作文总比赛\footnote{应征国家奖学金的比赛 . ——俄译者} 中甚至获得好评 . 但是教师们反对他升级 , 按照他们的意思 , 伽罗瓦的体格不够强壮 , 此外 , 中学校长认为他的判断力还有待 “成熟” . 但尽管如此 , 1826年10月他仍然转到修辞班\footnote{修辞班——中学的最高班 , 侧重于学习古代语言(拉丁语与希腊语) .  ——俄 译者}学习 .  但是第二学季\footnote{学季——英、美、法等国高等学校学年的一部分(三个月) . ——译者}一开始(伽罗瓦这时刚满十五岁) , 他不得不回到二年级 . 当时发生一件值得纪念的事情:埃瓦里斯特-伽罗瓦在 数学方面有了新发现 . 

在升入修辞班之前 , 中学全体学生都要按照下述的教学大纲 来学习:每个学生要修完中等学校必修的人文科学课程 . 但是学 生之中有些爱好精密科学的 , 也可以从二年级开始去听初级数学 的补充课程 . 伽罗瓦重修二年级 , 自然 , 他在这方面要比别人具有 更多的机会 . 他毫无阻碍地被批准去上数学课 . 

到了今天 , 人们毫无理由来设想 , 除了立志满足产生已久的迫 切的求知欲外 , 还有其他什么东西足以促使伽罗瓦坚持不懈地努 力下去 . 尽管他在新功课上向前猛进的速度看来不寻常 , 其实这 个现象不是无法说明的 . 只有与数学十分隔阂的人 , 才可能认为 ,  熟悉这一门科学是某种大彻大悟的结果 . 照此进行推论 , 那就等 于证明自己的无知 . 虽然在开学之时 , 全部数学教材的不平凡和 独具一格常常使一个学生惊讶不置 , 但是这种不平凡和独具一格 只是表面现象而已 . 至于伽罗瓦 , 那他在开步走之时就看透了这 些现象后面的推理的简单性 . 他懂得 , 在数学上掌握明确而富于 表达力的语言是何等重要的事 , 这正可证明他的思维的深刻 . 伽 罗瓦一开头就反对那些不谈推理方法而专谈引入迷途的技巧问题 的学校教科书 . 他不读这些教科书 , 而在几天之内一口气读完 A .  M . 勒让德尔\footnote{A . M . Legendre , 1752-1833 , 法国著名数学家 , 科学院院士 , 综合技术学校教 授 . ——译者}的经过多次重版(最后一版即第十五版于1881年出版)的经典著作《几何原理》(Eléments de Géométrie) . 勒让德 尔在他这本书里力求尽可能严格地、有根据地阐述当时已经被人 遗忘的欧几里德的八卷书 . 因此他必须回到欧几里德的推论方法 上面去 , 而把教师在几何学的课堂上教给他的一切都忘掉 . 勒让 德尔对欧几里德的不朽著作的修改主要是在叙述方注方面;但修 改的地方很多 , 事实上他的工作成果变成了几何学方面的一部崭 新的著作 . 伽罗瓦所领悟的勒让德尔的语言 , 本身包含着数学思 维的方法在内 . 

如果勒让德尔的几何学对于伽罗瓦讲来好比是一种新语言的 语法教科书 , 那末拉克朗日\footnote{J . L . Lagrange , 1736—1813 , 法国数学家 , 现代解析力学和纯解析的变分法的 创始人 . ——译者}的著作(《论数值方程解法》、《解析 函数论》、《函数演算讲义》)就起着语法的严格练习的作用 . 拉克 朗日所陈述的第一个问题 , 就使伽罗瓦有理由去应用他的“群” 概念 . 

这些艰深的功课当然还不足以显示伽罗瓦的独特的天才 . 但 是这些功课使他的思路开阔 , 而且很早就在他身上发展了科学家 所需要的惜以推测科学的主旨 , 而不是停留在枝节问题上的“预 见”的那种天赋 . 

因此 , 当1827年伽罗瓦回到修辞班时 , 他的全面发展甚至比 他的数学的天分在同学之中更加出人头地了 . 他并未失去对其他 科目的兴趣 , 但认为学校里讲授这些科目正如教科书里所讲述的 同样地潦草马虎 , 伽罗瓦对教师们所采用的教学法感到愤懑 . 但 在教师方面并不怀疑自己的学生有深刻的精神上的需求 . 有关这 一时期的一些笔记清楚地表明了由他所引起的混乱 . 有一个教师这样地谈到伽罗瓦:“他被数学的鬼魅迷住了心窍”;另一个教师用 了七个字:“平静会使他激怒”来形容他的行为 . 



这时候伽罗瓦已经熟悉欧拉、高斯和雅科比的著作\footnote{L . Euler , 1707-1783 , 士数学家 , 是数学史上最多产的数学家 , 其著作达886 种 . K . F . Gauss , 1777—1855 ,  德国数学家 .  K . G . Jacobi ,  1804—1851 , 德国数学家 .  伽罗瓦在1832年在失望之余曾企图把自己的理论就教于后二人 . ——译者} . 他很快就感觉到 , 他能够做到的 , 不会比他们少 . 伽罗瓦逐渐胆壮起来 了 . 到了学年末 , 他不再去听任何专业课了 , 他独立地准备参加取 得升入综合技术学校的资格的竞赛考试 . 伽罗瓦没有考取 . 不 过 , 尽管考试失败 , 1828年10月 , 他仍然从初级数学班跳到里查 (M . Richard)的数学专业班 . 

\begin{center}***
\end{center}

路易-勒-格兰中学的数学专业班教弗里查当时才三十三岁 .  从1821年起他已经当上数学教授 . 在科学史上 , 他作为一个很有 才华的教师而使人追念 . 在他培养下参加综合技术学校入学考试的人员当中 , 除埃瓦里斯特 \textbullet 伽罗瓦外 , 还有索尔奔纳(Sorbonne)\footnote{索尔奔纳—巴黎大学的一部分 , 建于1253年 , 以创始者罗伯 \textbullet 索尔奔纳 (1201-1274年 , 法国国王路易九世的忏悔牧师)的名字命名 . ——译者}天体力学教研室第一任主任、天文学家乌尔本 \textbullet 列 \textbullet 维利 叶\footnote{Urbain Le Verrier』811—1877 , 法国著名天文学家 , 曾预测海王星的存 在 . ——译者}和杰出的数学家查里士  \textbullet 厄米特\footnote{Charles Hermite ,  1822—1901 , 法国著名数学家 . ——译者} . 现在法国科学院图书馆 所藏的伽罗瓦的手稿 , 正是里查后来委托查里士 -厄米特保存的 . 

里查的学生们十分赞赏他讲课的优雅;他对科学工作具有独 特的风格 , 由他培养出来的综合技术学校的许多学生也有这种特 点 , 在很大程度上这也是他的功劳 . 发掘英才使里查感到莫大愉 快 . 伽罗瓦提出的算题解法 , 使他欢喜万分 . 他总是心满意足地 听着这个在他认为是自己学生当中最有天赋的孩子在同学们面前 讲话 . 里查遗留下的笔记既说明了老师 , 同时也说明了他的学生 的特点:“伽罗瓦只宜在数学的尖端领域中工作力” , “他大大地超过了全体同学” .  里查帮助伽罗瓦发表他的第一部著作并说服他向 科学院递送备忘录 . 伽罗瓦的文章发表在三月号的《数学年鉴》 (Annales de mathématiques)-这是热尔贡\footnote{J . D .  Gergonne ,  1771-1859 , 法国数学家 , 法国第一个数学杂志的创办人 .  ——译者}1818年所创办的法国第一个专业性的数学杂志 . 6月1日科学院举行会议 , 会上任 命普恩索\footnote{L . Poinsot , 1777-1859 , 法国科学院院士 , 综合技术学校教授 . ——译者}和科希\footnote{O . L . Cauchy , 1789-1857 , 法国伟大的数学家 , 和巴尔扎克同时代 , 其著作之 多 , 和巴尔扎克相似 , 但他作为正统派和保皇党 , 也和巴尔扎克的反动思想相似 .  ——译者}两人审查伽罗瓦寄来的著作 . 科希根本没有 作出任何结论;他丢掉了伽罗瓦的手稿 , 跟他以前丢掉阿贝尔\footnote{Niels Henrik Abel ,  1802—1829 , 挪威人 , 十九世纪最伟大的数学家之一 , 他 和伽罗瓦的数学思想非常接近 , 二人短寿相似 , 贫困潦倒相似 , 而手稿被人弄丢了的情 况也相似 . ——译者}的 手稿一样 . 

中学学年结束后 , 伽罗瓦在综合技术学校的入学考试中再次 遭到了失败 . 这是1829年的事 , 伽罗瓦当时刚满十八岁 . 里查以 及伽罗瓦的全体同学都惊讶不已 . 任何人也不怀疑这桩事件的后 果的严重性 . 发生这件事该怎么解释呢?伽罗瓦的天赋是不容置 疑的 . 因此要断定全部事情是由于行政当局的挑剔和主考人在品 评时的错误 , 看来是不可能的 . 结果势必认为 , 所以没有考取的过 错就在于伽罗瓦本人的不羁气质 . 有些人说 , “由于被提问所激 怒” , 他把黑板擦布往主考人头上扔;另外一些人说 , 他拒绝回答有 关对数的问题 , 他觉得这个问题过于简单 . 在监禁在圣佩拉吉监 狱的时期中 , 伽罗瓦提到这次考试时 , 写道 , 他不得不听“主考人的 狂笑声” . 这句话可以使人猜想到 , 当伽罗瓦阐述自己的见解时 , 有人居然嘲笑他 . 伽罗瓦的主考人是比内(Binet)和勒费布雷-德 \textbullet 富尔西(Lefebwe de Fourcy) . 比内毫无一点名气 , 至于勒费布雷 \textbullet 德 \textbullet 富尔西 , 却把他的无人问津的大量教科书堆满了图书馆的书架 . 他们给伽罗瓦评了多少分 , 无人知道 . 不管怎样 , 在 综合技术学校方面 , 伽罗瓦仍然是一个非正式的预备生 . 

倘若埃瓦里斯特-伽罗瓦考进综合技术学校 , 他就有非常优 越的条件 , 能够在两年期间使工作得到进展 . 当年 , 综合技术学校 的学生都有从事科研工作的可能 , 所以有大才干的人经常为此放 弃毕业后政府向他们提供的职务 . 很多综合技术学校的学生成为 卓越的数学家 , 从而使该校闻名全世界 . 现在情况变了 . 大资产 阶级尽力设法利用综合技术学校的学生为自己服务 , 截然不同于过去的任务也很能吸引着大学生们 . 他们在国民收入中所占的份额一代比一代增长 , 而数学家的培养则完全由别的学府所取代了\footnote{《数学简史》的作者 , 综合技术学校学生莫里斯 \textbullet 奥卡内(Maurice d'Ocagne) , 感叹综合技术学校里不再从事科研工作 , 他写道:“这样一来 , 综合技术学校回到从前它被称为‘国家高等工程学校’(Ecole centrale des travaux publics)时代所提出的任 务了 . ”谈到埃瓦里斯特 \textbullet 伽罗瓦时 , 莫里斯 \textbullet 奥卡内对伽罗瓦两次入学考试的失败避而不谈 . } . 

\begin{center}***
\end{center}

1829年7月2日 , 正当埃瓦里斯特准备入学考试的时候 , 他的父亲自杀了 . 这事发生在巴黎让-德-巴维(Jean-de-Beauvais)街上 , 那里有尼古拉-加布里埃尔 \textbullet 伽罗瓦的一所住宅 , 他来巴黎时就在那里歇脚 . 

事情的起因是布尔-拉-林市长在自己的城市里成了当地天主教教区牧师的攻击目标 . 年轻牧师以为 , 旧制度和宗教上偏执的时代又回来了 . 他不遗余力地迫害老伽罗瓦 , 不断地把匿名讽刺诗寄给他 , 而作者就是牧师本人 . 诽谤使老伽罗瓦生病 , 最后自杀 了 . 当老伽罗瓦遗体的送殡队伍来到布尔-拉-林市区的境界时 , 市民们从柩车上卸下棺材 , 用肩膀抬到墓地 . 牧师的出现引起了 冲突 , 结果牧师挨了一顿痛打 . 

伽罗瓦和母亲一起度过了服丧的日子 . 不管父亲的死使伽罗 瓦感到如何悲痛 , 他仍然“沉着而镇静” . 伽罗瓦听从里查的劝告 决定进师范大学 . 这使他有可能继续深造 , 同时使生活费用也有 个着落 . 随着丈夫的逝世 , 伽罗瓦的母亲失去很大的一部分收入 ,  而埃瓦里斯特还有一个十四岁的弟弟亚耳弗勒(Alfred Galois) . 

在1829年 , 师范大学(这里指的是它的预科)的情况与综合技 术学校毫不相同 . 师范大学是大革命后创建的 . 它要为高等和中 等学校培养师资 . 师范大学自开办以来经过了不少的变动 . 1822 年它遭到封闭 , 1826年以预科的名义重新成立 , 设有两个系:文学 系和科学系 . 学习期限为两年 .  1830年恢复师范大学旧称 , 同时 向学生宣布 , 学习期限延长为三年 . 国民教育检查官有否决学生 入学的权利 , 如果他认为学生的政治见解有嫌疑的话 . 伽罗瓦幸 免了这种遭遇 . 1829年10月25日他被录取入学 , 不过只算预备 生 . 最后录取要到1830年2月20日伽罗瓦签字同意为国家服务 六年并获得人文科学和自然科学毕业生的称号之后 , 才能正式 决定 . 

在1829年 , 师范大学的生活方式与修道院极为相同 . 吃饭前 和早课前后 , 全体学生都要大声朗读祈祷文;睡觉前必须听某一宗 教题目的谈话 . 每月要作一次忏悔 . 如果学生一连两个月一次也 不做忏悔的话 , 那就要被开除出校 . 校长亲自督促学生来遵行这 项规则 . 很多人指摘伽罗瓦行为古怪 , 性格执拗 , 但是 , 顺便说说 ,  他对忏悔的要求却做得很认真 . 留在师范大学很难使伽罗瓦感到愉快 , 然而这一年对他说来却是最顺利的一年 . 1829年 , 他的科 学研究获得了初次成果 . 伽罗瓦写了几篇大文章 , 并提出自己的全部著作来应征科学院的数学特奖 . 但在这里 , 他又遭到了新挫折:伽罗瓦的手稿原来交给科学院常任秘书傅立叶\footnote{Jean Baptiste Joseph Fourier4768—1830 , 法国著名的几何学家和物理家 ,  科学院院士 . ———译者} , 傅立叶收到 手稿以后不久就去世了 . 科学院不认为有必要通知伽罗瓦关于他的著作的遭遇 . 可是这些著作的某些抄本落到数学杂志《费律萨 克男爵通报》(Bulletin du baron Férussac)的杂志社手里 , 它在 1830年的4月号和6月号上把它刊载了出来 . 

在师范大学学习的第一年 , 伽罗瓦结识了奥古斯特 \textbullet 舍瓦烈(Auqust Chevalier) , 后者直到伽罗瓦临终前一直是他唯一亲近的朋友 . 舍瓦烈比伽罗瓦早一年考进师范大学 . 1830年10月 , 舍 瓦烈获得教师的称号 , 但他很快就退职了 . 奥古斯特 \textbullet 舍瓦烈是第一批坚定不移的圣西门主义者之一;他的兄弟米歇尔\footnote{Michel Chevalier , 1806—1879 , 法国资产阶级经济学家 , 在三十年代是圣西门主义的拥护者 , 1848年革命后 , 转到社会主义的敌方去 . 一译者} , 著名的经济学家 , 综合技术学校的学生 , 是这一运动的最初参加者之一 . 

当时围绕着圣西门的学说展开了激烈的辩论 . 伽罗瓦尽管信仰进步 , 但他并不附和圣西门主义者 . 他不理解“各尽所能 , 按劳付酬” 的口号中所包含的思想;他觉得这个公式似乎不够宽宏大量 . 虽 然青年人的狂热使他疏远圣西门主义 , 但是跟奥古斯特 \textbullet 舍瓦烈 的谈话却打开了他对当代政治问题的眼界 . 

\section{1830—1832年}
\begin{flushright}
	革命是全民族的事业 , 只有剥削人民的人除外 .
	
	戈德弗罗阿 \textbullet 卡芬雅克(Godefroy Cavaignac) 1831年
\end{flushright}

1830年 , 对自由党人说来 , 是巩固已有地位的一年 . 资产阶 级不断向欧洲各国政府献殷勤 , 忽而攻击左派 , 忽而攻击右派 , 同时把国家政权攫为己有 . 大家知道 , 这一过程早在拿破仑生前已 经开始 , 并且大大加速了他的垮台 . 1814年外国军队侵入法国 ,  它的标志便是国家有息证券的涨价和第一批大商行的出现 . 过不 多久 , 查理十世的集团已经完全依赖于银行了 . 1824年 , 政府不得不借几次债 , 其中就有英国某公司和拉斐德(Laffitte)银行的债款 . 1826年 , 资产阶级反对恢复引起当时出现大量领地的“长子继承权” . 1827年自由党反对实行出版法 , 因为这项法律威胁自由党的宣传自由 . 同时资产阶级尽力彻底丑化共和政体的思想 ,  因为不这样做就无法在国内维持秩序 . 这种政策自然受到正统 派\footnote{即波旁王族嫡系的拥护者 . ——译者}的赞许 . 这个政党的成员大多数由贵族组成 . 他们保存了自 己的财富 , 他们的利益完全和自由党人的利益一致 . 

在复辟期间 , 自由党不仅通过自己的代表所参加的国务院 , 而 且通过占据要位的自由党官员 , 来发挥自己的影响 . 大资产阶级 关心民族的利益大概比不上关心贵族的利益;至于无产阶级的利 益 , 那根本不在考虑之列 . 

人民对政治情况认识不清 . 大多数人的情绪是痛恨波旁王 族 , 人民认为波旁王族应对法国遭受的一切屈辱负责 . 波林雅 克\footnote{波林雅克公爵(J . A . Polignac ,  1780—1847)在1829—1830年担任查理十世的 首相和外交大臣 , 他的反动政策促使七月革命的到来 . ——译者}内阁组成后 , 资产阶级得出结论:查理十世不仅无用 , 反而有 害 . 预先准备好的机器开动起来了 . 党的领导和自由党人的受托 人路易-腓利浦\footnote{Louis-Philippe ,  1773-1850 , 是波旁王朝侧系奥尔良皇室的法兰西国王 ,  1830年登极 , 实行为金融贵族利益服务的反动政策 , 1848年革命后被迫退位 , 逃亡国 外 . ——译者}原先躲在幕后 , 现在粉墨登场了 . 路易-腓利浦 所住的皇宫成了生活的新主人们所经常往返的场所了 . 

1830年头几个月开始发行《国民日报》(Le National) . 为了迎合正统派寻求庇护的天然渴望而颁发的七月敕令\footnote{七月敕令——1830年7月26日查理十世颁发的四项法令:废除出版自由 , 解散下议院 , 召集9月6日和13日的选举人 , 以及新选举法 . ——译者} , 使自由党 人抓到所期待的发动斗争的借口 . 

商业巨头、工业企业和银行所有者——他们都是自由党的党 员——不能容忍他们的特权再次受到威胁 . 巧妙的宣传和人民生 活的贫困 , 保证自由党人获得左派的支持 . 有青年学生参加的共和派发动了人民;在巴黎 , 资产者在帽子上别上三色帽徽——七月革命\footnote{七月革命——法国1830年7月27日至29日进行的资产阶级革命 , 这次革命 推翻了查理十世的君主制度 , 建立了以奥尔良公爵路易-腓利浦为首的七月王朝 . 由于七月革命的结果 , 金融资本掌握了政权 . ——译者}开始了 . 

\begin{center}***
\end{center}

1830年7月 , 埃瓦里斯特 \textbullet 伽罗瓦将满十九岁 . 他在师范大 学的第一年功课行将结束 . 他在这时期写成的数学著作 , 已经使 人有可能对他的思想的独创性和敏锐性作出评价 . 至于政治方 面 , 当时还没有什么足以表明他有任何明确的立场 . 但是他对待 社会的态度发生那么迅速的进步 , 以致过了几个月 , 大多数具有自 由主义情绪的青年学生也远远落在他的后面 . 尽管这些青年(首 先是大学的学生和综合技术学校的学生)并不显得有多高的政治 觉悟 , 但他们仍有不少人参加了起义 . 只有师范大学的学生是个 例外 , 他们没有参加街上的示威 , 因为校长禁止他们上街 , 干脆把 校门锁上大锁 . 四十个学生中 , 只有两个人对这种措施表示愤慨 .  其中一个便是伽罗瓦 , 从7月28日夜晚到29日 , 他屡次试图溜到 街上去都未成功 . 这是他的第一次政治行动 . 

现在保存着有关一个很有趣的人物的若干资料 , 这个人便是 师范大学校长吉尼奥(M .  Guigniault) . 在教育部门的官员中间 , 他 算是唯一禁止自己学生参加示威游行的人 . 然而吉尼奥绝不是坚定不移地站在正统派原则上的热烈的保皇党;他根本不属于有足 够勇气坚持自己见解的一类人 . 他是一个性格脆弱 , 或者不客气 地说 , 是由于懦怯 , 总是站在胜利者一边的最平庸的自由主义者 .  1830年7月3日 , 当路易-腓利浦的胜利已是无可怀疑的时候 ,  在《地球报》(Le Globe)上出现一则报导 , 说是师范大学准备执行 新政府的命令 . 

吉尼奥本人也是师范大学的学生 , 1811年在该校希腊文化史 专业毕业 . 1818年他被任命为该校学生的课堂作业的领导人 .  到了 1830年 , 他已当上师范大学的教务主任兼校长 . 朱利 \textbullet 西蒙 (Jules Simon)在纪念师范大学一百周年的书中写道:“在吉尼奥 手下 , 人人战战兢兢 . 这个愚蠢、浅薄的人说起话来 , 总是神气十 足 , 而且在任何情况下 , 都保持着泰然自若的严肃态度 . ”
吉尼奥的官运和他的忠实朋友维克多-库申\footnote{Victor Cousin ,  1792— 1867 , 法国唯心主义哲学家 , 哲学上折衷主义学派的创立者 , 积极参加社会政治活动 . ——译者}的官运非常相 似 . 不论是吉尼奥 , 还是库申(他们两人对伽罗瓦被开除出师范大 学都负有责任)都是路易-腓利浦的驯顺奴仆 . 因为这个缘故 , 吉 尼奥才获得索尔奔纳教授的头衔 , 也因为同一个缘故 , 政府对维克 多 \textbullet 库申 , 也就是在1830年7月25日宣称白色旗子是国家可以 承认的唯一旗帜的维克多 \textbullet 库申 , 大施恩典 . 库申是师范大学的 学术委员会委员、索尔奔纳教授、皇家国民教育委员会的顾问、法 兰西贵族、国家特约顾问、法国科学院和道德政治科学院的院士 .  库申主要因为是当时有影响的、但目前已完全为人所忘记的哲学 派别的领袖而闻名 . 司汤达\footnote{Stendhal ,  1783— 1842 , 真名为 Marie Henri Beyle , 法国著名小说家 . ——译者}在《吕先 \textbullet 勒旺》(Lucien Leuwen) 一书中以寥寥数语而一针见血地说明了他的特征:“……具有高 贵思想而又居心狠毒的1829年型的自由主义者 . 他身居要职 , 每年可以从中获得四万法郎的收入 , 而他认为共和派是人类的耻 辱……” . 

\begin{center}***
\end{center}

自由党的力量获得增强并不是七月战斗的唯一结果 . 一小撮 出身资产阶级、但又鄙视本阶级的人 , 也精神振作起来 . 这些人自 称是共和派 . 1830年 , 他们还没有名副其实的政党 . 对现存制度 的反对态度使他们在思想上团结起来 , 而在组织上 , 他们联合成几 个爱国团体 , 其中最著名的是人民之友协会 . 国民议会是这些勇 士的理想 . 他们庄严地宣布 , 没有社会进步和社会福利 , 就没有未 来 . 在7月间 , 他们还不可能指望夺取政权 , 因为他们的队伍数量 太少 , 也还没有充分团结起来 , 他们以分散的小组参加战斗 . 拉 \textbullet  法耶特\footnote{M . J . P . La FayetteJ757~1834 , 法国大革命和美国独立战争的参加者 . 1830 年7月革命的国民自卫军司令 , 帮助路易-腓利浦获得政权 . ——译者}宣称:“当前局势的主宰者是共和派的政党 . 我们本来可 以轻而易举地取得我们的思想的胜利 , 但认为更明智的是把全体 法国人团结起来 , 在法国建立起自由而公正的立宪制度 , ”他是想 错了 . 戈德弗罗阿 \textbullet 卡芬雅克\footnote{Godefroy Cavaignac . l801-1845 , 法国著名政治家 , 共和派 , 积极参加反对 查理十世的君主制度以及以后反对路易-腓利浦的斗争 . ——译者}对局势的估计要现实得多 . 但是 在那时候 , 他回答一个自由主义者时却说:“你们用不着感谢我们 .  我们之所以让步 , 仅仅因为我们没有足够的力量” . 少数共和派不 得不满足于路易-腓利浦在市政厅所发表的声明 . 这些诺言并没 有兑现 . 路易-腓利浦政府忙于自己的卑鄙勾当 , 对防止发生混 乱的情况无能为力 . 7月发生饥荒 . 大臣杜潘(Dupin)在贵族院 宣布 , 在十个工业区里 , 一万名适龄应征入伍的人当中 , 有八千一 百八十人不适宜服兵役 . 工厂里越来越广泛地使用童工 , 选举权 的资格限制没有废除 . 至于对外政策 , 它比对内政策更加辜负了 共和派的期望 . 当时曾任驻伦敦大使的塔烈兰(Talleyrand)千方百计地努力与法国邻邦保持着和睦的关系 . 当时曾经签订了若干 秘密协定:与西班牙签订的是法国承担义务向西班牙通知有关西 班牙逃亡法国难民中的叛乱情绪;与俄国签订的是把暴乱的波兰 委付给沙皇;与普鲁士签订的是要普鲁士严防青年德意志派的活 动;与奥地利签订的是听任奥地利在意大利恢复被美诺蒂\footnote{Menotti , 意大利革命家 , 縻地那城的烧炭党领袖 . ——译者}所动 摇的旧制度 . 因此 , 在法兰西的帮助下 , 欧洲的几次革命运动被镇 压下去了 , 这班革命运动的领导人都是坚定地期待着1830年7月 推翻波旁君主制度的人们来给予援助的 . 路易-腓利浦的对外政 策是蹂蹒各民族的利益 , 对内政策是与人民的利益背道而驰 . 在 法国 , 直到最近 , 对内实行公民自由和对外尊重国家主权是相互密 切地联系在一起的 . 伽罗瓦很知道君主政体和人民利益的不相 容;他经常使用“爱国者”一词来代替“共和派人” 或者反过来使 用 . 

共和派的威信在7月还是毫不足道的 , 可是到了 11月 , 就不 能不予重视了 . 路易-腓利浦的政策使许多人惶惶不安 . 政府对 不满情绪的增长不再是充耳不闻了 , 报刊上开始了反对共和派言 论的运动 , 这班人无非是所谓“过激分子” , 他们之中最积极的就受 到了警察的监视 , 有几个情报员被暗中派到人民之友协会里去 , 在 开头的时候 , 有几宗挑拨离间的事件被人发觉了 . 共和党的影响 虽然日益增长 , 实际上还是很容易被击破的 . 共和党的领袖们只 相信一种美德——勇敢 . 他们尽管估计到有人民群众的支持 , 但 并不注意广泛宣传自己的思想 . 他们的斗争策略极为简单 , 不外 是散发传单 , 号召仿效国民议会的榜样 . 政府就亳不迟疑地利用 了这些失策和错误以利于己 . 

\begin{center}***
\end{center}

1830年10月 , 埃瓦里斯特 \textbullet 伽罗瓦回到师范大学学习 . 很难说 , 他的共和政体的信念最初是在什么时候流露出来的 . 不论 是他自己 , 还是他的亲友 , 都不曾留下任何有关他如何度过1830 年前夜的真实材料 . 不错 , 在他死后六十年 , 他的一个亲属曾经力 言:在和自己悒郁不欢的家庭——他使用的正是“悒郁不欢”一词 ——谈话时 , 埃瓦里斯特 \textbullet 伽罗瓦热烈地捍卫人民的权利 . 但无 论如何 , 我们现在既然无法怀疑他的洞察力和意志力 , 我们就不难 设想 , 他之决定追随共和派是一桩多么勇敢而又多么慷慨的行为 .  这个面色苍白、神情忧郁的青年总是出现在大无畏者的行列中间 .  无怪乎他的科学著作的特点首先也是思想上的大胆独创 . 早已失 去青年热情和青年血气的那种具有自由主义情绪的年轻人的短暂 热情 , 是与他格格不相入的 . 未来——这才使他真正感到兴趣 .  有一次他谈起某些学者时说道:“这班人落后了一百年 . ”

伽罗瓦加入人民之友协会显然是在1830年11月10日之后 ,  因为他是按照当时新制定的章程被吸收入会的 . 这个章程是:“凡 愿意加入本会之公民须经两名会员介绍 , 并在入会申请书上共同 签名 . 申请书须送交中央干事会审查 . 决议采取秘密投票方式 .  如有两张反对票 , 就取消候补会员资格……书面讨论应予禁止 . ” 这些预防措施是为防止奸细钻入协会而采取的 . 

在加入人民之友协会的同时 , 伽罗瓦报名参加了国民自卫军 炮兵队 , 其中有两个炮兵连完全由共和派分子组成 . 

在师范大学中 , 伽罗瓦是唯一参加人民之友协会的学生 , 因 此 , 他自然不局限于只向自己同学解释共和党的纲领 . 伽罗瓦开 始攻击师范大学的领导人 , 也就是攻击上述那个校长吉尼奥和上 述那个教授库申 . 

当时 , 库申和吉尼奥都是查理十世的君主立宪政体的热烈拥 护者 , 而且给《地球报》撰稿 . 后来他们两个人摇身一变 , 当了路 易-腓利浦的忠实走狗 , 成为通常称为大学的封建世袭领地的重要领主 , 钻进支持新制度的特殊人物的派系里去 . 师范大学的学 生们并不把他们的这些变化看作可耻的现象 , 反而一举一动都极 力仿效他们的领导人 , 认为这才使他们有前途 . 伽罗瓦很鄙视吉 尼奥在7月日子里所表现的“谨慎”同样鄙视他过后全面改变自 己的见解 . 除了政治上的动机外 , 还有对师范大学教育组织的不 满 . 他所能听到的对于自己的全部异议的唯一答复就是自古以来 人们都知道的一句话:好学生不过问政治 . 同学们也不赞同伽罗 瓦的行动 . 他显得很孤立 , 甚至当吉尼奥不定期地把他软禁在家
里的时候 , 他依然是孤零零的 . 这种处罚办法 , 撇开其他方面不 谈 , 主要是剥夺伽罗瓦和他的共和派朋友们见面的机会 . 他不能屈服于这种手段 , 因此决定立即反抗 . 在伽罗瓦悲惨的一生中 , 这是切断他的一切后路的步骤 . 伽罗瓦知道得很清楚 , 如果他公开让人知道自己的事业 , 期待着他的会是什么 . 这无疑是意味着“从事政治”而且是名副其实的 , 尤其是站在维克多-库申所视为人类耻辱的共和派的一方面 . 在伽罗瓦这样热情洋溢、胸怀坦荡的青年看来 , 采取这个决定和他的科学发明同样重要 . 埃瓦里斯特 \textbullet 伽罗瓦逝世已经一百多年了 , 可是至今人们还不肯宽恕他的这桩 事情 . 

\begin{center}***
\end{center}

三十年代出版了主要以科学界人士为对象的两家报纸 . 其中 一家《中学报》(Le Lycee)热烈赞扬现状 , 并保护1830年7月以前高据要津的官员们不受科学界的攻击 . 不过 , 必须指出 , 辞职的 人一般并不多 , 科希为了不向路易-腓利浦宣誓而辞职要算最重要的事件了 . 吉尼奥和库申是《中学报》的撰稿人 . 另一家《学校 公报》(La Gazette des Ecoles)提出了 一个广泛的纲领 , 这个纲领 在其纲要中提出了这样的要求:“团结起来 , 为1793年的伟大改革 而奋斗!完成我们时代的使命——已经开始的改造”这家报纸在本质上捍卫一小群对新制度心怀不满的官员 . 

《学校公报》经常提到师范大学校长的名字 . 他跟伽罗瓦发生 的纠纷 , 给予这家报纸以新的借口再一次对他进行攻击 . 在1830 年12月5日该报出版的星期日报上 , 发表了一篇长文章 , 它的作 者批评师范大学的领导 . \footnote{见本书第三部第三节文件 . }似乎为了证实自己的说法 , 文章援引一 封署名“师范大学一学生”的信 , 信中嘲笑吉尼奥在7月日子里的 行为 , 而且特别强调他的机会主义 . 一般都认为写这封信的是伽 罗瓦 . 他没有正面肯定这种看法 , 同时也没有加以否认 , 尽管这封 信的语调一点也不吻合他惯常的风格 . 无论如何 , 伽罗瓦跟发表 这篇被编辑修改得便于在激烈的争论中加以利用的文章有一些瓜 葛 . 从报纸方面说 , 这当然很失策 , 正如企图以匿名信隐瞒文章作 者的真姓名一样地失策 . 以这种方式向读者进行揭露 , 尖锐性就 差得多了 , 但是编辑部却利用伽罗瓦缺乏经验 , 把全部责任转嫁在 他身上 . 几个星期后 , 就是这一家《学校公报》发表了反对伽罗瓦 的文章 . 利用伽罗瓦一说就越发真实有据了 . 

文章发表四天之后 , 即12月9日(星期四) , 吉尼奥指示把伽 罗瓦送回家去 , 尽管伽罗瓦的罪过还没有被证实 , 吉尼奥却向教育 大臣作了报告 . 

吉尼奥在报告中称伽罗瓦为懒汉和道德败坏的青年 . 他一口 咬定 , 开除伽罗瓦可以使师范大学和整个巴黎学区摆脱害群之马 的牵累 . 现在看来 , 这篇极端不高明的声明实在叫人感到不胜 惊奇 . 

但是 , “第一所新型高等学府的首脑”(吉尼奥这样自称)并非 满脑子都是“远离政治”思想的单纯的傻瓜 . 他还是一个懦夫 . 因 为担心不能很轻易地摆脱伽罗瓦 , 他就企图唆使师范大学学生进 行告密 . 赶走伽罗瓦以后 , 他开始搜集揭露“罪犯”行为的材料 . 

由于吉尼奥同他的学生进行多次谈话(既然学生们的未来掌握在 他的手里 , 他们对他的威胁就不能置若罔闻)的结果 , 十四个文学 系的学生 , 连名给《学校公报》寄去一封指摘伽罗瓦的信 . 科学系 的学生只附上比较冷静的、干巴巴的附言 . 伽罗瓦自己中止了这 场笔战 , 向师范大学发出了一封公开信 . 他干脆而又慎重地警告 他的同学们要注意到他们是受人怂恿而作出一些不名誉的勾 当的 . 

1831年1月8日 , 皇家国民教育委员会批准了这一开除 处分 . 

“根据库申顾问先生有关暂时开除伽罗瓦的报告 , 并注意到师 范大学校长叙明采取该项措施的原因的报告 , 兹决定:

立即将伽罗瓦开除出师范大学 , 以资儆戒 . 

有关他今后前途的决定容后另行议处 . ”

出自维克多 \textbullet 库申之手的这一决议的草稿 , 一直保存到 现在 . 

\begin{center}***
\end{center}

1830年12月 , 因为共和党日益增长的影响而感到惶恐不安 的政府 , 组织了第一次但又十分狡猾的挑拨行为 . 

12月8日本杰明 \textbullet 贡斯当\footnote{Benjamin Constant , 1767—1830 , 法国政治家 , 自由派 , 写过很多有关国家制 度问题的文章 . ——译者}逝世 . 他死于贫困 , 但因为自由 主义者的政党在很多方面有赖于他 , 政府决定为他举行隆重葬礼 .  综合技术学校和师范大学的学生也应邀参加送殡 . 路易-腓利浦 想通过人民的盛大集会 , 引诱社会舆论不去注意行将举行对查理 十世的大臣——波林雅克倒台内阁的阁员的诉讼案 . 这件诉讼案 原定于12月15日在改为法庭的贵族院开庭审判 , 但在本杰明 \textbullet  贡斯当出葬的日子里发生的骚动 , 此起彼落 , 有增无已 . 

无论是路易-腓利浦本人 , 还是他的大臣们 , 都不愿判处被告以死刑 . 但是他们不能忘记 , 人民认为波林雅克以及跟他合伙的 人是人民全部灾难的祸首 . 他们只好玩弄手腕 . 首先应当把事情 办得使罪犯可以避免死刑 . 这样路易-腓利浦才能在欧洲各国政 府心目中保持威信 , 并赋予自己的王朝以合法的性质——这一点 他看得高于一切 . 甚至这次为保全大臣们的生命而作的决定引起 了人民的小骚动也在所不惜 , 因为随之而进行的镇压 , 既可以使法 兰西与欧洲言归于好 , 同时又可平息自由主义者的情绪 . 路易-腓利浦在进行一场赌博 , 而且赌赢了 . 

12月21日 , 贵族院判决大臣们终身监禁 . 在前一天 , 先把犯 人移到维桑内(Vincennes)城堡 . 内务大臣宣布 , 这次迁移的目的是为了保护囚犯躲开愤怒的人民 . 果然不出所料 , 被告的缺席缓 和了紧张气氛 . 现在再来对付人民就不难了 . 国民自卫军和大学生都在政府控制之下 . 综合技术学校大开校门 , 学生队伍挤满巴 黎街头 , 号召人民保持镇静 . 根据国民自卫军总司令拉 \textbullet 法埃特 的命令(他害怕在行将到来的事件的风暴中会名誉扫地) , 自卫军 军人都模仿大学生的榜样 . 工人们自7月巷战以后还记得这些制 服款式 , 这时受到制服的蒙蔽 , 他们就分散回家去了 . 12月23日 政府对学生和国民自卫军表示感谢 . 但过了几天 , 借口改组 , 把国 民自卫军解散 , 拉-法埃特也被免除了总司令的职务 , 这便是第二 部分的预谋计划 . 只有两个营拒绝解除武装 . 结果十九名炮手被 逮捕 , 共和党反对派有个时期就大大削弱了 . 

在这种局势下 , 没有一个教师 , 没有一个科学界活动家敢于对 针对“共和派”伽罗瓦的措施提出异议 . 何况对某些人说来 , 这意 味着排除一个危险的对手 , 而在另一些人看来 , 这是犯政治错误的 公正处分 . 同时大家都一致认为 , 如果有一个人成为众人的异己 分子 , 而又不重视自己圈子里的规章 , 那么 , 毫无疑问 , 应该把他驱逐出校 . 只有《宪法报》(Le Constitutionnel)将师范大学学生伽罗 瓦的遭遇告诉自己的读者 . 

维克多-库申和他的帮凶吉尼奥共同策划的把伽罗瓦开除出 师范大学的处分 , 除了其他后果外 , 还使伽罗瓦失去生活费用 .  1831年1月9日星期日 , 《学校公报》刊登了下列一则奇怪的声明:

“1月18日(星期四) , 伽罗瓦先生将讲授高等代数 . 以后每 逢星期四下午1时15分 , 将在索尔奔纳街第5号 , 凯洛特(Ca illot)小书铺进行讲课 . 该讲座以不满足专科学校所讲授的代数 而希望深造者为对象 . 讲座将向听众介绍不曾公开讲授过的若干 理论 . 其中某些理论完全是独创的 . 只须列举新虚数论、用根式 求解的方程论、数论和用素代数研究的椭圆函数论便足以概见全 豹 . ”

第一次讲座在预定日期的指定钟点准时进行 , 课堂内有三十 名听讲人 . 当代科学史上还不曾有一个年轻科学家(伽罗瓦这时 刚满十九岁)肯向广大听众阐述自己新颖和独创的思想借以谋生 .  显然伽罗瓦确实具有罕见的坚强个性 . 

\begin{center}***
\end{center}

1831年1月17日举行的科学院例会上 , 两位科学院院士:拉克鲁阿\footnote{S . F . Lacroix , 1765—1843 , 法国数学家 . ——译者}和泊松\footnote{S . D . Poisson , 1781—1840 , 法国著名数学家 , 科学院院士 . ——译者}受委审查伽罗瓦的科学研究简述 , 简述的手稿业 经伽罗瓦在例会前夕送交科学院秘书处 . 这份简述的原作全文在 一年前已呈送科学院 . 当时这份全文是在常任秘书傅立叶手上 ,  后来傅立叶来不及加以研究就去世了 . 他死后遗留下来的文件中 没有发现这份全文 , 由于第二次呈送自己的著作 , 伽罗瓦曾附上一 篇简短绪言 , 要求“至少”要仔细地读完他写的东西 . 这种坚持看 来绝不是多余的 , 因为要不是伽罗瓦给科学院院长写了一封很不客气的信 , 他的著作也就不会在科学院例会上宣读了 . 伽罗瓦在 信中初次说出自己的推测 , 认为人们对他的著作固执地保持沉默 ,  是和他的名誉所受到的影响有关 . 

因此 , 想一想艾米尔 \textbullet 皮卡尔 \footnote{Emile Picard , 1856-1941 , 法国数学家 . 从1917年起担任法国科学院常任秘书 . ——译者} 为1897年出版的伽罗瓦作品 的初版所作序言 , 倒是很有趣的 . 他写道:“尽管这事令人伤心 , 却 使人产生一种印象 , 仿佛这位不幸的青年要以新的不幸为自己的 每一个天才的发现而付出代价似的 . 随着数学家伽罗瓦的杰出才 能越来越显露 , 一向为人纯朴而乐观的伽罗瓦的处世态度却越来越阴郁了 . 与日俱增的优越感发展了他的漫无节制的自豪感 . ”创 造这段无稽之谈的荣誉当然并不属于艾米尔 \textbullet 皮卡尔 . 他所写的 不过反映一种广泛流行的看法 . 伽罗瓦看到他的功绩不受人充分重视时 , “极端的自豪感”促成他进行反抗 , 同时也使他没有可能成为社会上的一个平等成员 . 同一社会在另一种情况下是很可能接纳他甚至对他表示尊敬的 . 不能不令人同意的是 , 伽罗瓦产生这种情绪是有充分的理由的 . 投考综合技术学校的失败、向科学院 提出两份研究报告的遗失、父亲悲惨的自杀一难道这还不算多 吗?这些理由越发有分量 , 是因为它既能将以往事件的责任推到 伽罗瓦本人身上 , 又消除了错在他人的最微小的怀疑 . 这些理由 只有一点不足之处 , 那就是它们都是谎话 . 不存在着两个伽罗瓦 . 数学家伽罗瓦和共和派伽罗瓦是同一个人 . 如果了解伽罗瓦的数学著作 , 甚至最不成熟的读者也会感觉到 , 他的著作中的一切都是朝向未来的 . 伽罗瓦谈到“未来数学家的使命” , 谈到他“所选择的 道路” . 就是这个伽罗瓦在一次政治的诉讼上宣布:“我们是小孩 ,  但是 , 我们精力充沛 , 勇往直前 . ”

\begin{center}***
\end{center}

1831年4月的头几天 , 开始对国民自卫军的几名炮手提出诉 讼 . 在塞纳省陪审法庭前 , 站着1830年国民自卫军被解散后拒绝 解除武装的十九名年轻人当中的十六位青年 . 

市政府警卫队占据了司法宫的走廊 . 上流社会青年挤满厢 座 , 学生和工人拥挤在法庭大厅的门边 . 被告在辩护律师——象 他们一样地也是共和派——的陪同下进入大厅 . 当他们出现时 , 大厅里响起了欢迎的呼声 . 自从1830年7月以来 , 共和派得不到 一次适当的机会来宣传自己的观点 . 所以这时候被告们没有想到 辩护 . 相反 , 他们在进攻 . 他们之中有些人说到大城市平民生活的骇人听闻的贫困 , 另一些人揭露他们称之为背叛革命原则的现 象 . 作为证人的戈德弗罗阿 \textbullet 卡芬雅克 , 则大谈共和党的纲领 .  他断言 , 传播共和思想的事业无须秘密活动 . 因为“革命是全民族 的事业 , 只有剥削人民的人除外;这是我们祖国的事业 , 它正履行 着民意所赋予它的解放使命;这是向人类克尽己责的法兰西事业 .  至于我们 , 先生们 , ”在结束自己的讲话时 , 他大声疾呼道 , “我们是 革命的奴仆 . 我们时刻响应着革命的号召 . ”律师们很轻易地证明 , 有关策划旨在以共和政体代替君主制 度的秘密阴谋的指控是毫无根据的 . 全体被告都被宣判无罪 . 

同日晚上 , 在巴黎许多房屋里 , 闪耀着节日的彩灯 , 为了适当 地庆祝所取得的胜利 , 人民之友协会5月9日在郊区坦普尔\footnote{Temple , 巴黎的城堡、塔楼与教堂 , 建于十三世纪 . 塔楼在18世纪末资产阶 级革命时期成为国家监狱 . ——译者}的 “布尔根饭店”举行盛大宴会 . 在荣誉席上 , 协会中央理事会成员当中 , 坐着亚力山大 \textbullet 仲马\footnote{Alexandre Dumas ,  1803—1870 , 即大仲马 , 法国著名小说家 . ——译者} , 坐在他身边的是尤贝尔特(Hubert)、 马拉斯特\footnote{A . Marrast , 1801—1852 , 法国政论家和政治家 , ——译者}和腊斯拜 . 非常漂亮的年轻人彼歇-德-艾尔宾维尔 (Pescheux d'Herbinville)也在场 , 关于他 , 仲马曾经说 , 他做的主要事情就是用绢纸做蝇拍 , 并用玫瑰色绦带来装饰拍子 . 在应邀 出席宴会的二百名爱国者中间 , 也有埃瓦里斯特 \textbullet 伽罗瓦 . 为了 避免跟警察发生冲突 , 举杯致词是预先准备好的 , 并且约定不发表 任何演说 . 但宴会的组织者忽略了一群年纪最轻而热情奔放的共 和派可能因为自己的领袖们对讲演所采取的敷衍态度而感到愤宴会快结束时 , 其中一个不满者即席举杯致词 , 只说了一句 话:“为路易-腓利浦干杯”他一手举杯 , 一手持刀 . 这个人就是 埃瓦里斯特-伽罗瓦 . 大多数与会者报以热烈的掌声 , 而没有看 见刀子的人则表示抗议 . 坐在荣誉席上的宴会组织者慌张起来 .  亚力山大 \textbullet 仲马和他的一个朋友、皇家剧院的演员一起立即越窗 而逃 . 宴会结束时 , 根本谈不上什么秩序了\footnote{由于这次事件 , 五个月前曾经捍卫过埃瓦里斯特 \textbullet 伽罗瓦的《学校公报》 , 现在开始反对他 . 下面是发表在5月12日号上的一篇简讯:“……有很多人举杯祝酒 , 这 时有个狂人怒气冲天 , 忿然离席 , 从衣袋拔出刀子 , 在空中挥舞起来 , 高声喊道;“我要 向路易-腓利浦宣誓……”这个“狂人”就是埃瓦里斯特 \textbullet 伽罗瓦 . } . 

第二天早上 , 伽罗瓦在他母亲的屋里被逮捕 , 在进行侦查的全 部期间内 , 他被关进圣佩拉吉监狱 . 人民之友协会试图通过该会 的律师说服伽罗瓦放弃他说过的话 . 但所有这些努力都属徒劳无益 . 

6月15日在塞纳省陪审法庭上开始审查案件 . 伽罗瓦被控教唆谋害法兰西国王的人身和生命的未遂罪 , 虽然此举并未发生任何后果 . 

本书引用这次开庭的报告\footnote{见第三部第四节资料 . }——完全保持6月16日号《辩论 杂志》(Journal des Débats)刊载的原文——绝非出于对其中细节的生动描写的喜爱 . 叙述者的诚实态度和鲜明的叙述文体使这篇简讯成为有关共和党活动和埃瓦里斯特 \textbullet 伽罗瓦的独特性格的珍贵文件 . 

被告席上坐着一个身体柔弱、精神活泼而充满自尊心的少年 .  他简短而尖酸地回答法庭庭长的问题 , 有时迅速地说出一两句热 烈的激动人心的话 , 却因话带讽刺而让听众感到轻松 . 他 , 这个被 告 , 机智伶俐 , 什么也逃不过他的注意 . 谈到政治时 , 他就利用政 治文件 . 至于他是一位数学家 , 那倒没有什么意义 . 在确定伽罗 瓦身分的预审中 , 他满不在乎地说 , 他在“帮人补习数学” .  顺便说 一句 , 这时候他在索尔奔纳街的公开课已经完全停止了 . 

多亏通常充当共和党人的律师窦本(Dupont)的努力 , 伽罗瓦 被宣告无罪 , 当场获释 . 

\begin{center}***
\end{center}

7月11日 , 政府通过了逮捕共和党领导人的决议 . 与此同 时 , 在米(Mie)印刷厂里纪念7月14日国庆节的号召书全部被没 收 . 这份告巴黎人民的呼吁书写道:

\begin{center}
	\emph{7月14日国庆节}
	
	活动程序
\end{center}

“月14日(星期四) , 爱国者们集合在巴士的狱\footnote{巴士的狱(Bastille)——十四世纪至十八世纪巴黎的城堡 , 同时也是国家监狱 . 自十五世纪起 , 主要监禁政治犯 . 1789年7月14日由于人民起义而被捣毁 , 这次起义奠定了十八世纪末法国资产阶级的革命 . 自1789年 , 捣毁巴士的狱这一天定为法国国庆纪念日 . ——译者}广场上集合并栽植自由之树 , 以纪念攻克巴士的狱和法兰西共和国成立四十 三年周年 . 

“夏特勒(Châtelet)广场和花堤岸街的集会在中午举行 , 示威 游行将在一点钟开始 . 游行路线:堤岸街、圣马丁街、街心公园、巴 士的狱广场 . 

“自由之树将由7月战斗的参加者组成的仪仗队护送 . 游行行列由演奏爱国歌曲的军乐队引导 . 装饰着花条绦带和三色绦带的树枝由1789年的老战士和在‘伟大的一周\footnote{“伟大的一周”——指1830年7月27日武装起义开始和同年8月2日查理十 世退位之间的七天 . ——译者}'中负伤的战士高 擎着 . 

“工人、学生、7月革命的参加者、资产阶级出身的青年以及一 切爱国人士都被邀请参加这次庆祝会 . 自愿参加庆祝的国民自卫 军军人请穿制服出场 . ”

\par

吓破了胆的政府当局禁止示威游行 , 警察继续逮捕共和派人 .  7月13日至14日夜间 , 大多数及时得到预先通知的人民之友协 会的会员都没有在家过夜 . 这救了伽罗瓦 , 他当时住在柏纳丁斯 (Bernardins)街 . 伽罗瓦从自己的共和派朋友那里得到指示后 ,  于7月14日中午到新桥去 , 并与一个法律系的学生杜沙特列 (Duchâtelet) 一起率领六百名示威者 , 警察不费力气地把这两个 领头人和示威群众分隔开来 , 并抓走他们 . 在这里提到杜沙特列 的名字并非偶然 . 几乎可以肯定 , 在1832年5月30日的决斗中 他就是伽罗瓦的对手 . 

两个被逮捕的人关在道芬(Dauphine)街警察局所属的预审拘 留所里 , 但在同天晚上他们被迁移到圣佩拉吉监狱 . 示威游行在7月14日持续了一整天 . 晚上在爱丽舍宫广场(Champs-Ely-sées) , 共和派受到预先由警察局乔装成“工人”的市政府警卫队的 袭击 . 第二天报纸上发表了被逮捕的最著名爱国者的名字:杜布 尔(Dubourg)将军、杜富尔(Dufour)将军和所谓“年轻的伽 罗瓦” . 

伽罗瓦从1831年7月14日起至1832年3月16日 , 被关在圣佩拉吉 . 他在这里庆祝过他的二十岁生日 . 也在这里 , 他得悉 , 

早在7月11日科学院的例会上 , 他在1月12日呈送科学院审查 并在3月31日的信中重新提到的研究报告 , 被否定了 , 科学院援 引泊松和拉克鲁阿的结论拒绝肯定伽罗瓦阐述的论点的正确性 . 

“……泊松先生不愿意或者不能理解 , ”后来伽罗瓦本人说到这事时这样写着 . 

不论是内务大臣 , 还是警察局长 , 都清楚知道他们的这位新犯 人对共和党的功劳;他的数学天赋对他们也不是秘密 , 正因为如 此 , 他们对待他才特别严厉 . 在开始审查这件案子之前 , 已经过去了不少时间 . 到了 1831年10月23日 , 也就是被逮捕三个月零九日以后 , 伽罗瓦和杜沙特列才出庭受审 . 为了避免经过陪审法院又一道诉讼程序(在陪审法院可能作出宣告被告无罪的判决) , 对被告只提出非法穿军服和携带武器的控诉 . 在被捕的时候 , 伽罗瓦和杜沙特列穿着国民自卫军炮兵的制服 , 携带马枪 . 此外 , 在搜查伽罗瓦时 , 发现暗藏一把匕首 . 杜沙特列被判处三个月徒刑 , 伽罗瓦则被判处九个月徒刑 . 十分清楚 , 这种区别是不能光用从伽罗瓦身上搜出匕首一事加以解释的;显然 , 上述区别有一定作用 .  伽罗瓦对判决提出上诉 , 但巴黎法院1831年12月3日作出的最 后判决 , 维持上述判决有效 . 在判决书里特别强调一点 , 即不论是伽罗瓦 , 还是杜沙特列 , 都没有权利穿国民自卫军的炮兵制服 , 因 为1830年改组自卫军以后 , 他们两人已经不是自卫军的成员 . 

\begin{center}***
\end{center}

关于圣佩拉吉监狱的情况现在还保存有足够的资料 . 大家都 知道 , 在这所监狱里 , 犯人分为三类:政治犯、刑事犯(包括因欠债 而被监禁的人)和未成年犯 . 孩子们处境最为恶劣 . 至于政治 犯——正统派、拿破仑派 , 而主要是当时被大批逮捕的共和派人 ,  他们占据着设备最完善的一部分房子 , 他们本身又分为三种类型 .  最有钱有势者住单房间 , 生活费用自备 , 从邻近的饭店购买膳食 . 

比较年轻而比较不重要的人物 , 七八人合住一间房 , 不过享有同样 的特权 . 穷人则住在每六十人合住一间房的牢房里 . 夜间全体共 和党犯人参加他们称之为“晚祷”的仪式——唱“马赛曲”和“进行 曲” .  做完这种“祈祷”之后 , 开始演戏剧 . 通常是表演比喻七月革 命事件的节目 . 大栅栏权当必要的布景 , 而用作演员道具的只有 一样东西——装着被路易-腓利浦杀害的共和国遗体的棺材 . 演 出进行到午夜一点钟 . 白天 , 大多数政治犯把时间消磨在监狱院 子里所附设的小酒馆中 . 1831年在圣佩拉吉监狱喝掉的烧酒可 真不少 . 对体质羸弱、经常深思熟虑的伽罗瓦说来 , 这所监狱谈不 上是个“安静的住所” . 

在1832年2月初大搜捕中被捕的热拉尔 \textbullet 德 \textbullet 内尔瓦尔\footnote{Gérard de Nerval , 1808—1855 , 法国著名浪漫主义作家 . ——译者}在《我的监狱生活》一书中 , 叙述他在圣佩拉吉监狱中住过几天的 生活 . 在政治犯当中 , 伽罗瓦是他唯一记得住名字的人 . 

“我正和我的一大堆难友愉快地吃午饭 , 这时有人在楼梯上喊 道:‘热拉尔 \textbullet 德 \textbullet 内尔瓦尔 , 收拾行李和武器 . ’这意味着我释放 了 . 我非常欢喜圣佩拉吉监狱中的生活 , 甚至再留下一天我也是 心甘情愿的 , 然而我不得不离开 . 我想至少吃完午饭才走 , 但这也 办不到 . 不久竟然出现一种奇怪的场面:犯人被强制离开监狱 .  当时是五点钟 , 有个同桌吃饭的人把我送到门口 , 吻了我一下 , 答 应他一旦获释 , 就来拜访 . 他自己必须再坐两三月牢 . 他就是不 幸的伽罗瓦 . 可是我从此再也见不着他了 , 他出狱的第二天就被 杀害了 . ”
这一友谊的供述不仅说明个人的倾慕 , 而且说明两人精神需 求的接近 . 

在那几个月里 , 腊斯拜是伽罗瓦的狱中难友 . 他和伽罗瓦不 同 , 虽然不享受任何特权 , 但在圣佩拉吉监狱中占有一个单房间 , 因此他有更大的可能进行工作 . 在他的《寄自巴黎监狱的书简》中 有若干涉及伽罗瓦这个时期生活的资料 . 尽管腊斯拜有时被认为 具有“伟大的灵魂” , 但他的思想和它的表达形式常有粗糙的缺点 .  不过《书简》一书中的个别意见 , 可使人清晰地想象到迫不得已生 活在腊斯拜这类人当中的伽罗瓦所怀的忧郁绝望的心情 . 有一 天 , 有人约伽罗瓦打赌一人喝一瓶烧酒 . 伽罗瓦接受了这个挑战 .  后果是很可怕的 . 腊斯拜对发生的事情表示遗憾 , 他写道:“宽容 这位柔弱而大无畏的少年吧!三年之中 , 科学在他额上划下六十 年深思熟虑也不会更深的皱纹 . 为了科学和德行 , 请爱护他的性 命吧!再过三年 , 他必将成为真正的科学家 . ”腊斯拜忘记写的只 是:他本人并没有做任何事情来减轻这个受他如此热烈拥护的人 的遭遇 . 

伽罗瓦在监禁期间还是不停地工作 . 看来 , 他想在获释后立 即写出两部著作 . 奥古斯特 \textbullet 舍瓦烈在他这位朋友死后加以整理 的文件中 , 发现有两份笔记显然是这两部著作的序言 . 在其中一 篇笔记里 , 伽罗瓦攻击科学院的成员 , 特别攻击泊松 . 攻击得非常 激烈 , 以致第一个出版伽罗瓦手稿的朱利-汤内里不敢公布这篇 笔记 . 本书登载了它 . 十分明显 , 伽罗瓦有足够愤怒的理由 , 我们 认为 , 隐藏他写的任何一篇文章都是错误的 . 

\begin{center}***
\end{center}

1832年3月16日 , 伽罗瓦因病从圣佩拉吉监狱转移到设在 努尔申(I'Oursine)街第86号的医院里 . 医院受警察局监督 , 是由一个叫做福耳特里埃(Faultrier)领导的 . 很可能 , 除直属职务外 ,  福耳特里埃还兼做情报工作 , 正因为如此 , 他才担负起监视病人的 责任 . 有一种说法 , 伽罗瓦在4月29日在他的监禁期满以后 , 还 在这里逗留了一些时候 . 这所医院是我们所知的关于伽罗瓦最后 的居住地点 . 非常可惜 , 在努尔申街的房子里几乎没有留下伽罗瓦居住过的痕迹 , 而他在4月29日以后的生活又是情况不明 , 令 人无从推测 . 5月30日他离开家 , 去参加决斗一一这是大家确实 知道的全部的真实情况 . 

我们知道伽罗瓦这一时期的一些生活情况应当归功于当时住在孟尼尔蒙坦(Ménilmontant)的奥古斯特 \textbullet 舍瓦烈 . 这里 , 在圣西门公社里 , 奥古斯特-舍瓦烈、他的兄弟密歇尔以及其他许多 人 , 按照他们的“导师”巴扎尔和昂方坦\footnote{圣西门去世后 , 阿尔曼 \textbullet 巴扎尔和普罗斯贝尔-昂方坦【À . Bazard , 1791-1832; B . P . Enfantin ,  1796 -1864 , 法国空想社会主义者 , 前 者是圣西门主义的卓越宣传家 , 后者是该组织的最主要的奠基人 . ——译者】是他的事业的最积 极继承者 . 巴扎尔在孟尼尔蒙坦组织农业公社 , 昂方坦到埃及去参加尼罗河上游冰坝 的建设 . 兴建水坝是为了使这个国家恢复从前的肥沃 . }的原则过着安闲恬静的 生活 , 奥古斯特 \textbullet 舍瓦烈好几次劝说自己的朋友与他分享田园式 生活的快乐 , 但伽罗瓦坚决地加以拒绝 . 

在决斗三个月后发表出来的文章中 , 奥古斯特援引自己朋友 的一封后来引起很多反应的信 . 这封信的热烈而激动人心的感情 很难使人无动于衷 . 但尽管如此 , 首先使人感到震惊的 , 并非他的 激昂情绪 , 而是抑压着这位青年的无限疲倦 . 多数评论者不能宽 恕伽罗瓦的这句话:“憎恨!只有憎恨!”如果他们想起一一即使已 经迟了!一一他为科学所做的事 , 而这点已经为大家所公认 , 他们 就容易理解到 , 他是应当怀着憎恨或者热爱的感情的 . 但是 , 他们 却忘记了作为科学家的伽罗瓦 , 而情愿把他的感情一古脑儿算在 作为一个人的伽罗瓦的账上 . 

总之 , 伽罗瓦获释了 . 他指望在6月初离开巴黎 . 在给奥古 斯特 \textbullet 舍瓦烈的一封信中 , 他承认“要在一月之内尽情享受一个被 释放的人的甜蜜的快乐……”事实是 , 伽罗瓦在福耳特里埃的家中 遇见一个造成5月30日决斗原因的女人 . 关于这个女人 , 人们什 么也不知道 . 有些人怀疑 , 她是遵照警察局的指示行事的 . 不过 , 既然我们认为伽罗瓦不是跟彼歇-德-艾尔宾维尔决斗 , 如亚力山大 \textbullet 仲马所断定的那样 , 而是跟1831年7月14日在新桥和他一起被捕的武装战友杜沙特列决斗 , 我们觉得这种假设并没有根据 . 伽罗瓦在一封信里说得很清楚 , 他的对手是一个爱国者 . 

很难找到比伽罗瓦临死前所流露的内心感情更高尚的榜样了 . 5月29日 , 即在决斗前一天 , 他写了三封著名的信:一封给共和派的同志们 , 一封给N . L , 和V . D .  , 而最出色的是给奥古斯特 \textbullet 舍瓦烈\footnote{见第三部第一节文件 . }的信 , 这封信的大部分是谈数学问题 . 伽罗瓦死后 , 在他的桌子上发现两张纸条 , 其中一张现在还可以看出这么一句话:“这 个论据需要补充 . 现在没有时间 . ”日期是:“1832年” . 显然 , 在 临决斗前 , 他还在校正这些数学分析的著作 . 

5月30日清晨 , 在冈提勒的葛拉塞尔湖附近 , 伽罗瓦受了致命伤 . 决斗双方用手枪在相距几公尺的地方互相射击 . 一颗子弹 击中伽罗瓦的腹部 . 几小时之后 , 当地一个居民偶然发现了他 , 就把他送到科申医院 . 

“不要哭 , ”伽罗瓦对最后几分钟和自己在一起的弟弟亚耳弗 勒说 , “不要哭 , 我在二十岁的年纪死去 , 需要我全部的勇气 . ”伽罗 瓦拒绝了神父替他祈祷 . 

1832年5月31日上午10时 , 伽罗瓦与世长辞 . 

\begin{center}***
\end{center}

巴黎的几家报纸转载同样的一篇简讯报导伽罗瓦之死 . 这篇 简讯是按巴黎警察局局长吉斯克(Gisquet)的指写的;他认为伽 罗瓦是“有影响的共和派”(他在自己的回忆录里写了这一点)并害 怕他的葬礼会成为骚动的导火线 . 地方报就较有可能写得真实 ,  例如里昂的自由主义报纸《先驱报》(Précurseur)在6月4日一期中 登载了一篇报导:


“巴黎6月1日消息 . 昨天 , 一场不幸的决斗从科学界夺走了一位前途极其灿烂的青年 . 可叹他过早的声望仅仅和政治有关一年前 , 由于在“布尔根饭店”宴会上曾经举杯致词而遭法院审讯的年轻的埃瓦里斯特 \textbullet 伽罗瓦 , 跟自己的-位年轻朋友进行了决斗 . 两位年轻人都是人民之友协会的会员 , 两人都曾在政治诉讼案中出庭受审 . 有消息说 , 决斗起因是由于恋爱事件 . 决斗双方挑选手枪作为决斗武器 . 他们从前曾经是朋友 , 因此认为不屑互相瞄射 , 而决定听天由命 . 他们用枪口互相顶着对方射击 , 但其中只有一支手枪装着子弹 , 子弹射穿了伽罗瓦 , 他被抬到科申医院 , 两小时以后就在那里死去 , 伽罗瓦足年二十二岁 , 他的对手L . D . 比较稍为年轻一点” . 

除了年龄上的差错以外 , 这篇文章写得完全真实 . 和伽罗瓦在政治诉讼案中一起出庭的只有一个共和派人——杜沙特列 , 这与姓名第一字母“D”完全符合 , 这些新的细节使得所谓挑拨离间 的假设变得很成问题 . 

伽罗瓦于1832年6月2日星期六安葬 . 

“今日中午举行埃瓦里斯特 \textbullet 伽罗瓦的葬礼 . 送葬的有人民 之友协会的代表团、法律系和医学系的大学生、巴黎炮兵部队以及 他的许多生前好友 . 当送殡行列来到郊区公园时 , 从柩车上卸下 灵柩 , 由人抬到蒙帕尔纳斯公墓 . 普拉尼奥尔(Plagniol)和查理 士 \textbullet 彼涅尔(Charles Pinel)两位市民致悼词 , 深切地表达了死者 的朋友们的悲痛 . 有两位爱国者也以同样的方式悼念埃瓦里斯 特 \textbullet 伽罗瓦”(1832年6月3日《运动论坛》[La Tribune du Mouve-menty]) . 

1832年9月 , 奥古斯特 \textbullet 舍瓦烈在《百科全书派评论》 (Revue Encyclopédique)上发表了悼念自己的故友的文章 . 在这以后 , 埃瓦 里斯特 \textbullet 伽罗瓦的名字长期被人遗忘了 . 伽罗瓦的全部数学著作
从他的弟弟亚耳弗勒-伽罗瓦的手里转到奥古斯特-舍瓦烈手 上 , 但他找不到愿意出版这部著作的人 . 到了1846年 , 著名学者 约瑟夫 \textbullet 刘维\footnote{Joseph Liouville , 1809 -1882 , 法国数学家 . ——译者}才第一次在他创办的数学杂志上发表了这些 著作 . 

在这以前 , 埃瓦里斯特 \textbullet 伽罗瓦的同时代人开始忘记了他 . 

某些人有意识地不去想他 , 免得增添麻烦 . 

有一个年轻人 , 由于政治信念坚强 , 曾经特别受到埃瓦里斯特 \textbullet 伽罗瓦的尊敬 , 但也叛变了伽罗瓦 , 因为这样一来 , 对自己的 前途不无好处 . 

六十页手稿向世界宣布了科学家伽罗瓦的名字 . 从此以后 ,  他的天才开始在科学上名列前茅 . 单凭公道 , 就要求我们即便是现在 , 也应当对这位天赋如此非凡的人在短促一生所经受的种种痛苦 , 寄予同情 . 




