\chapter{埃瓦里斯特 \textbullet 伽罗瓦 和科学的发展}


\begin{flushright}
	理解——就是承受并继续已开始的东西 .
	
	让 \textbullet 卡韦累斯(Jean Cavaillés)
	
	“我在这里进行分析之分析”
	
	埃瓦里斯特 \textbullet 伽罗瓦
\end{flushright}


伽罗瓦的数学著作 , 至少是那些保存下来的 , 共有六十小页 .  从来还没有象篇幅这样小的著作曾经给作者带来如此之高的 声誉 . 

要了解伽罗瓦所做的事 , 需要一番特殊的努力 . 伽罗瓦对烦 琐累赘的计算方法感到不可抑制的厌恶 , 因此 , 他的表述极其简单 扼要 . 但是他所写的一切 , 都因为具有数学家不倦钻研的思想而 放出异彩;他的每一部著作 , 仿佛都是一次新的大胆的跃进;先前 已达到的又落在后面 , 不再使作者发生兴趣了 . 伽罗瓦的洞察力 是惊人的 . 他对待读者的态度有时似乎很傲慢(他并不怎么关心 读者的兴趣) , 但实际上这不过是思想十分独特、具有坚定的目的 性的例证而已 . 

尽管伽罗瓦大力研究高次方程理论 , 但他并不单纯是一位杰 出的代数学家 . 他对他所得出的具体结果 , 从来不给予很高的评 价 . 首先使伽罗瓦感到兴趣的 , 并不是个别的数学习题 , 而是决定 一连串想法的亦即指导思维运动的论证方法或“方式” . 他的论证 方法的基础是一个能够概括当时已经达到的成就并决定科学长期 向前发展的深刻理论 . 伽罗瓦去世几十年后 , 德国数学家大卫 \textbullet 希尔柏特\footnote{David Hilbert , 1862—1943 , 德国著名数学家 , ——译者}把这种理论称为“一个明确的概念结构之建立”(l'établissement d'une certaine charpente de concepts) . 但不 管给它 安上什么名称 , 显然它是包括很广泛的知识领域的 . 

很多从前彼此孤立地加以研究的各种理论 , 事实上只不过是还需要做精密计算和实际应用的个别情况而已 . 所以 , 数学家就不必去进行计算;正如伽罗瓦所说的 , 他们只要“预见到”如何计算就行了 . 一百二十五年前\footnote{此句系俄译者加的 , 本书出版于1956年 , 伽罗瓦的研究报告写在1831年 , 中 间相距一百二十五年 . ——译者}在圣佩拉吉监狱写成的研究报告中 ,  就对这一点处处作了明白的说明 . 任何一个认真的人 , 即使他与 数学毫无关系 , 也不能不感觉到贯穿在字里行间的热烈的信念的:“……总之 , 我以为 , 依靠改进计算而获得的简化(这里当然是 指原则上的简化 , 而不是指技术上的简化)绝对不是无穷无尽的 .  终有一天 , 数学家必将能够如此清楚地对代数变换作出预见 , 以致 不必再花时间和纸张来认真进行计算 . 我并非断言 , 除了这种预 见外 , 分析不能有其他的新成就 , 但我认为 , 如果没有这种辅助工 具 , 有朝一日 , 全部分析方法都将成为徒劳无益的东西 . 

使计算听命于自己的意志 , 把数学运算归类 , 学会按照难易程 度 , 而不是按照它们的外部特征加以分类——这就是我所理解的 未来数学家的任务 , 这就是我所要走的道路 . 

但愿任何人都不要把我流露出来的急躁情绪跟某些数学家向 来对无论哪一种计算都要根本回避的意图混为一谈 . 他们不用代 数公式 , 而使用冗长的议论 , 在重迭的数学变换之上 , 又加上对这 些变换的重迭的文字概述 , 所使用的又是不适于解答这些算题的 语言 . 这些数学家落后了一百年 . 

在这里没有这类的情况 . 我在这里进行分析之分析 . 与此同时 , 现在已知的最复杂的变换(椭圆函数) , 只不过被看作十分有益 的甚至是必不可少的、但究竟是个别的情况 , 因此拒绝作进一步 的、更广泛的探讨 , 将是一种不可挽救的错误 . 总有一天 , 在这个 粗具轮廓的高等分析中所提到的变换 , 将真正地得到实现 , 并且是 按照难易的程度 , 而不是按照这里出现的函数形式予以分类” . \footnote{见第三部第二节 . }

长期以来 , 没有人知道伽罗瓦在1832年拟定的计划是否真有 其事 . 这一计划在他逝世七十年后才发表出来 , 可是在当时它仍 然不引起人们多大的兴趣 , 不久也就被人忘记了 . 只有我们这一 代年轻的数学家们 , 由于继承了好几代科学家的工作 , 最终 , 才实 现了伽罗瓦的理想 . 然而 , 正是他的著作 , 标志着数学前史的结束 和数学史的开始 . 

虽然伽罗瓦的科学活动惊人地短促 , 现在仍然可以观察到他 是怎样逐步地得出如此深刻的结论的 . 在前面刚刚援引的摘录 里 , 读者应该注意到“把数学运算归类”这一句话 . 毋庸置疑 , 这是 指目前称之为群论 , 即从十九世纪末叶开始 , 对数学分析、几何学、 力学而最终是物理学的发展有着巨大影响的群论而言 . 创立这个 理论的荣誉属于埃瓦里斯特-伽罗瓦 , 他是第一个估计到这个理 论对科学前途的意义的 . 这就是他为什么很想略为说明 , 即便是 非常扼要地、但仍然十分明确地说明他所进行的工作的实质 . 

伽罗瓦所研究的问题之一 , 长期以来吸引着数学家们的注意 .  这就是解代数方程的问题 . 我们每个人在学校学习时 , 都要解一 次和二次方程 . 解方程 , 意即求出它的根值 . 但在三次方程中 , 这 点决不容易办到 . 伽罗瓦研究的是任意次方程 , 即方程的最一般 情况 . 

顺便一提 , 从实践观点看来 , 无论形式多么复杂的任何具体方 程的解并没有任何意义 . 早在十六世纪 , 数学家就已经发现 , 使用能确定方程根的近似值的方法较为便当 . 这些近似值充分满足了 物理学家、化学家和工程师的需要 . 目前采用计算机来计算 , 可以 毫不费力地得出任意精确的结果 . 但对于使用字母作系数的一般 方程讲来 , 近似法是求不出它的根值的 . 我们每个人都可以拿起 一张纸 , 记下这样的一般方程 , 并用特定字母表示它的根 . 这些根 当然是未知数 . 伽罗瓦的第一个发明就在于他把这些根值的不定 式的次数减低下来 , 也就是确定这些根的某些特征 . 伽罗瓦的第 二个发明就是他所使用的求得这个结果的“方法” . 伽罗瓦不研究 方程本身 , 而研究它的“群” , 或者比方说 , 研究它的“家族” . 

群的概念是在伽罗瓦著作提出之前不久才出现的 . 但是在他 的那个时代 , 它却象是一个没有灵魂的躯体 , 是偶尔出现在数学上 的、人为臆断的大量概念之一 . 伽罗瓦的革命性 , 不仅在于他使这 个理论具有生命 , 他的独创性又赋予这个理论以必要的完整性;伽 罗瓦还指出 , 这一理论富有成效 , 他并且把它运用到解代数方程的具体习题上 . 正因为如此 , 埃瓦里斯特 \textbullet 伽罗瓦是群论的真正创 始人 . 

群是具有某种共同特性的对象的总和 , 譬如可用实数作为这 些对象 . 实数群的共同特性在于 , 如果令群中的任何两个元相乘 ,  则其积仍为实数 . 几何学所研究的平面位移可以代替实数作为 “对象”;在这种情况下 , 群的性质表现为任何两个位移之和就是一 个新的位移 . 从简单实例转到复杂实例时 , 我们可以选择关于某 些对象的运算自身作为“对象”在这种情况下 , 群的主要特征将 表现为 , 任意两种运算的结合也是一种运算 , 伽罗瓦的研究正是 这样 . 在分析求解的方程时 , 他把某种运算群与这个方程联系起 来(遗憾的是 , 我们在这里无法更明确地说明这点是怎样做到的) ,  并证明方程的特性反映在该群的特点上 . 既然不同的方程可以 “有”(avoir)同一个群 , 那么 , 无须研究所有这些方程 , 只须研究与之相适应的群就可以了 . 这一发现标志着数学发展的现阶段的
开始 . 

不论群是由什么“对象”——数、位移或运算——组成 , 它们全部可以看成不具有任何特征的抽象的对象 . 要测定群 , 只须说 明为了使某“对象”的总和可以称为群而应加遵循的共同规则就可以了 . 目前数学家们把这些规则称为群的公理 , 群论是依据这些公理运用逻辑的结果 . 新的特性连续不断地发现出来了;在证实这些特性时 , 数学家也使这一理论越来越得到发展 . 极其重要的 是 , 不论是对象本身 , 还是对它们的运算 , 都不是具体的 . 要想研 究任何特殊问题 , 那就必须分析组成群的某些专门数学的、或是物 理的对象 , 然后 , 根据一般理论就可预见到它们的特性 . 由此可 知 , 群论显然是经济地处理问题的“方法”;此外 , 群论又为研究工 作提供了新的数学工具 . 

“我恳求我的评判人至少读完这几页”——伽罗瓦在他的著名的研究报告中的开端这样写着 . 如果他的评判人足够正直无私 , 那我们就会原谅他们缺乏敏锐的眼光 , 因为伽罗瓦的思想如此深邃 , 如此广阔 , 以致当时无论哪一位学者确实很难估量他的思想的价值 . 

\begin{center}***
\end{center}

很多思想家执拗地试图确定 , 天才究竟是什么 . 这种尝试 , 其 结果都将徒劳无益 , 因为这种天分一向被看作与产生它的环境没有关系的一种空想的现象 . 实际上 , 巴斯加尔\footnote{B . Pascal , 1623—1662 , 法国数学家 , 物理学家 . ——译者}的天才 , 比方说 ,  并不在于他十二岁的时候就能重演欧几里德几何学中的头三十二道命题 , 甚至也不在于他结识得扎尔\footnote{G . Desargues 1593—1662 , 法国数学家 . ——译者}之后就写了一部论圆锥曲线的著作 . 巴斯加尔的天才在于他发现了各门不同学科之间新的、过去不为人们所知的联系:“但愿人家不要说 , 我不曾作出什么 新发现 . 新发现在于拥有材料 . 两个人在玩棒球戏时 , 双方使用 的是同一只球 . 但其中一人找到了对他更好的位置 . ”(巴斯加尔:《思维》的序言)这位研究者首先发现的不是一些新对象 , 而是它们 之间的联系 . 

一旦没有必要 , 天才就会消声匿迹 . 这种想法是很容易证实 的 , 只要在人们企图表明政治家同一般从事政治的人有什么区别 时 , 把通常谈到的那些关于政治家的话推广到科学家身上就行了 .  政治家头一个发觉世界各种力量对比中所发生的变化;他头一个 意识到行动的必要性 , 并且据此来选择行动的方式 . 科学界的情 况也是如此 . 当有必要发生某种根本性的变化时 , 科学家的天才 就流露出来了 . 人类知识的发展过程是不平衡的 . 有时候某一部 门的进展暂时中断 . 科学在停滞中 , 昏昏欲睡 . 科学家们从事琐 碎的事情 , 把贫乏的思想隐藏在华丽的计算后面 . 十九世纪初期 ,  代数变换已变得如此复杂 , 以致向前进展实际上已经成为不可能 的事情了 . 笛卡尔(Descartes)想出的并经他的拥护者加以改进的 数学工具打消了原来建造这种工具的目的 . 数学家们不再能够 “预见”了 . 甚至拉格朗日也无法把搁浅的解代数方程的问题向前 推进(伽罗瓦却做到了这点) . 拉格朗日的无能为力是当时代数学 衰落的鲜明例证 . 必须寻找新道路的时机来到了 . 决定这一时机 的决不是偶然性 , 而是必然性 . 天才的特征就在于能觉察到这种 必然性 , 并且立即对它产生反应 . 

伽罗瓦写道:“在数学中 , 正如在任何其他科学中一样 , 有一些 需要在这一时代求得解决的问题 . 这是一些吸引先进思想家思想 而不以他们个人的意志和意识为转移的迫切问题 . ”

科学史上留下一些学者的名字 , 这些学者由于特别爱好钻研 的精神而能及时感觉到刻不容缓的紧要变化 , 并且向自己的同时代人指出了这一点 . 科学也高度尊重那些善于实现必要变革的 人 . 有时候(尽管这是罕见的现象) , 一个人可以兼具这两点 . 拉 瓦锡就是这样的人 , 埃瓦里斯特 \textbullet 伽罗瓦也是这样的人 . 

在这里提到拉瓦锡的名字并非偶然 . 十八世纪下半叶化学的 发展暂时停顿下来了 . 天才的化学家仍然是足够多的 , 化学实验 的技术也达到了尽善尽美的地步 , 以致当时的许多成就一直沿用 到现在 , 但科学却停留在原地不动 . 拉瓦锡首先注意到术语不够 明确 , 而且千篇一律 . 当化学著作中充满定律和概念上的混乱时 ,  根本不可能有什么进展 . 以拉瓦锡的著作为起点 , 化学开始了新 的繁荣时期 . 

在某种意义上 , 伽罗瓦在数学上的贡献和拉瓦锡在化学上的贡献一样 . 群的概念的建立 , 使数学家们摆脱了研究大量的、各式各样的理论的繁重负担 . 原来人们只要指出这些理论是可能的就行了 . 由于这些理论就其本质而言都是十分类似的 , 所以用同样 一句话就足以表白它们 . 伽罗瓦说:“我在这里进行分析之分析” ,  这种想法表明了他竭力想使这些新的、象辞汇表那样地具有实用 意义的基元得到使用 . 群论首先是数学语言的整理 . 

巴斯加尔的“新排列法” (disposition nouvelles)、拉瓦锡的“化学命名法”(nomenclature)、伽罗瓦的“群”——所有这些出色的发 现 , 一再表明 , 确定新联系在科学上会起着多大的作用 . 其中每一 项发现都标志着科学家所使用的语言的重要改进 . 

\begin{center}***
\end{center}

凡是谴责伽罗瓦的政治活动 , 或者干脆不考虑他的政治活动的人 , 都不能认清他在科学上所作的贡献的价值 . 其所以不能 , 是因为他们认为 , 理论似乎与实践无关 , 只有具体活动似乎才是正经事 , 而任何一般的推论只是无聊的游戏 . 对他们说来 , 进步是隅然 的事 , 而发明则是奇迹的结果 . 这班人认为 , 科学家的工作是超时间和超空间的 , 他在某种抽象的世界中生活 , 进行创造 . 这种观点 很便当 , 它使人无所用心 . 

埃瓦里斯特 \textbullet 伽罗瓦反对科学家的天生孤独性 , 并为此付出 了生命 . 除了他自己以外 , 这是谁的过错呢?为了缓和这种生硬的说法 , 人们曾经想出一种特殊的解释:说伽罗瓦过分年轻 , 说他过激了、但同时却忘记了他的头脑是惊人地清醒的 . 

埃瓦里斯特 \textbullet 伽罗瓦早在圣佩拉吉监狱里 , 就想到未来科学家们的团结:“科学家生来并不比其他人更要过孤独的生活的;他们也是属于特定时代的人 , 而且迟早要协同合作 . 到了那时候 , 将有多少时间腾出来用于科学!”

大概 , 没有一位科学家有过象埃瓦里斯特-伽罗瓦那样把科学理想与社会理想结合起来的;大概 , 这种结合从来不曾引起来自 政府方面如此疯狂的迫害的 . 