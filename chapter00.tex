\chapter*{引言}


\markboth{Introduction}{引言}

\begin{note}
	
	阅读整理原书,在书后附整理了一次、二次、三次、四次方程解法的一点资料,做为参考。
	
	本资料仅为个人学习、排版。
	
	如需要代码、指导、交流,可发邮件至:hnkznhb@126.com。
\end{note}

\par

1832年5月30日星期五早晨,有个农民在冈提勒(Gentilly) 的葛拉塞尔(Glacière)湖附近看见一个陌生人昏迷不省地躺在地 上.后来发现他是在用短枪决斗后受了重伤被遗弃在这里的.人 们把这个不知名的受伤者抬到科申(Cochin)医院.第二天早上十 点钟他就死了.

行年二十岁的埃瓦里斯特 \textbullet 伽罗瓦——全世界学者迄今公认 的、曾有特殊功绩的、卓越的数学家,就这样地断送了生命.

伽罗瓦——法兰西科学之光,在他的著作中体现了法兰西科 学的优秀特点,他的死使数学的发展推迟了好几十年.

伽罗瓦的短暂的一生充满着惊人的事件.当他还是路易-勒-格兰(Louis-le-Grand)中学的学生时,他就发表了他的第一部著 作.三年以后,因为积极参加政治生活,他被开除出了师范大学. 热情洋溢的共和党人伽罗瓦曾经两度入狱;他在决斗前还把最后 的时光献给整理数学论文的工作.所有这一切都不能不使写文章 论述他的人寄予同情,立意为这个具有非凡才华、在政治斗争的曲 径上迷途的不幸的少年人写一部传略.有些人甚至认为埃瓦里斯 特 \textbullet 伽罗瓦之所以产生暴力革命的思想,是由于个人遭受到许多 挫折,使他的自尊心时时受到鞭挞的结果,而他的与痛恨旧制度有 关的政治见解则是由于他个人性情乖戾所致.但是,不管这幅画 像多么饶有浪漫色彩,骤然看来它又是多么合乎情理,我们还是把 它丢开为妙.事实上,这位数学家的命运是比人们对他的理解更 加合乎规律,他的失败和挫折并非偶然之事.不应该随便把埃瓦里斯特 \textbullet 伽罗瓦的生活与他的时代的重大事件任意地割裂开来, 传说纷纭,最终,不但以讹传讹,而且将造成违反常识的差错.埃 瓦里斯特 \textbullet 伽罗瓦的一生经历完全可以证实上述那些说法是不妥 当的.

资产阶级想到一个有天才的人居然会参加人民的进步运动, 就很难容忍.一个学者要出人头地,首先得证明自己无害于人. 假使他一开头就并非没有害处,资产阶级会力图使他变成害群之 马.这就是为什么一个学者必须避免所谓“参加政治力的原因.这样的说法,意思就是说,他必须避免参加支持资产阶级反对者的政 治活动.因为显而易见(或者一般人认为是显而易见),任何不满 情绪的表现都会妨碍科学的发展!

埃瓦里斯特 \textbullet 伽罗瓦的最后一封信是以这两句话结束的:《别 了!我为公共的福利已经献出了自己的大部分的生命”. 埃瓦里斯特 \textbullet 伽罗瓦诞生在拿破仑帝国时代,经历了波旁王朝复辟的时 期,又赶上路易 \textbullet 腓利浦朝代初期.他眼看资产阶级(他就是这个 阶级的子弟)抛弃社会正义和社会福利的思想,并且随着政治上的 摇摆不定,忽而向左、忽而向右地寻求支持.伽罗瓦是在当时最先 进的政治集团即共和党的行列中进行斗争的.当时的共和党是革命者的政党.这些共和党人认为,公民的平等权利和平等义务是 社会正义的基础,追求社会正义的渴望应该是进步的实质.对进 步的热烈信念在很多方面决定了伽罗瓦的工作.数学家伽罗瓦的 优点和革命者伽罗瓦的积极性,是他热爱这种崇高思想的两种 表现.

为了证实上述说法,我还要指出,构成数学创作的那种日常工 作是不可能在忙碌与杂乱之中进行的.没有经常性的工作,数学 家埃瓦里斯特 \textbullet 伽罗瓦就不可能存在.因此倡言伽罗瓦过激,就 意味着忘记他是处在青年时期中并且抹煞了他的记忆能力.当他在综合技术学校的入学考试中完全出乎意料地遭到失败时,他的 一个中学同学这样写道在交卷以后,他可以毫不怀疑:他将被录 取.可以想象得到他的心境.但是,尽管伤心,他仍然沉着而冷 静.”让我们记住这句话:“尽管伤心,他仍然沉着而冷静.”

这本书,是我们献给埃瓦里斯特 \textbullet 伽罗瓦以表示尊敬的,因为 他虽然年轻,但在数学和政治上却大有成就.然而,如果把埃瓦里 斯特 \textbullet 伽罗瓦的功绩简单地归结为不寻常的早熟,那就没有比这 更可恶、更卑鄙的了.伽罗瓦不是神童.他生前并不出名.他的 同时代数学家们不仅不懂得伽罗瓦的著作标志着数学发展的新时代,甚至不重视他的著作.必须经过半个世纪以后,科学界才认 清他的思维独到之处和深刻的程度.但是,现在也很少有人认识 到,伽罗瓦所特有的预见才能不仅表现在数学上,而且还表现在他 对当时的“社会名流集团力的批判和他跟这种集团的斗争上.假使 伽罗瓦一生中没有如此激动人心的事件,那么人们一般都很乐意 忘掉他这方面的天才.我们却与一般的见解不同,我们认为吸引 他参加这种生活的,绝不是他对冒险的爱好,而是内心强烈的激 情.埃瓦里斯特临死六天前给他的朋友写出下面的话并不是偶 然的:“我违背理智地感到内心愤懑;但是我并不象你那样补充说: ‘非常遗憾'.”

\begin{center}
	***
\end{center}

本书包括三个部分.第一部分专谈伽罗瓦的生平.埃瓦里斯 特 \textbullet 伽罗瓦的传记第一次发表在师范大学1896年的年鉴上;1903年佩吉(Péguy)在《半月回忆》(Cahier de la Quinzaine)第五集第 二期上重新加以转载.这篇传记的作者杜普伊(Dupuy)在搜集资 料方面下了很大工夫,除开一些文件外,他还获得伽罗瓦同时代人 口述的许多材料和数学家的亲戚们所谈的一些回忆.令人惋惜的
是,有关伽罗瓦私生活的细节恰好是这本写得虽然极为认真、但又 过分宽容的著作中的最大弱点.结果是,那些牺牲了伽罗瓦的高 尚的荣誉心以求个人明哲保身的人都得到了开脱,然而杜普伊文 章中所包含的事实材料,一般说来是确实可靠的,尽管他采用的各 种材料并非自始至终都是准确的,甚至当这些材料的作者都是数 学家时,也是如此.

至于本书,我们力求尽先以埃瓦里斯特 \textbullet 伽罗瓦生活的历史时期为背景来描写他,为此我们采纳了某些新文件,其中之一报导 了 1832年5月30日决斗的详细情况.

第二部分是试图说明伽罗瓦在科学发展中的作用.我们并无 沽名钓誉之意,并不妄图补充学者们已经给他的著作所增补的学 术注释.使我们感到兴趣的,并不是专门的科学问题,而是伽罗瓦 谈到科学组织的新体系和科学家之间必须实行合作的、通常为人 们所忽略的个别见解.读者将为这些文章的感染力和现实性感到 惊异.但是,姑且不论伽罗瓦提到的问题,就连他的语言直到现在 也没有人加以研究,这实在不能不令人感到惊奇,尽管拉瓦锡\footnote{A.L.Lavoisier,1743—1794,法国著名化学家.——译者}早 已说过,科学家的语言本身,就是一套完整的方法.

在第三部分中汇集了一些文件.我们觉得这部分最重要,因 为伽罗瓦的书信和真本笔记使它具有特殊意义.当然,这里不包 括早已收入专版书中的数学著作,但是伽罗瓦所写的其余全部著 作,包括关于他在塞纳省陪审院法庭上的诉讼报告,当时报刊上的 文章、传记和其他收入第三部分的材料,在这里要比在任何地方都 齐全.






