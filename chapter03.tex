\chapter{资料}

本书这一部分收入两类资料 . 第一类是伽罗瓦本人写的全部 文件 , 但不包括他的数学著作 . 其中几篇原文过去只片断发表过 ,  或者从未发表过 , 第二类文件是说明伽罗瓦生平若干事件的 材料 . 

1)\emph{埃瓦里斯特 \textbullet 伽罗瓦的书信} \quad 这一部分包括七封信 . 其 中一封 , 《论数学教育》 , 自1831年以来 , 不曾再版过 . 大概 , 是因 为第一次发表这封信的《学校公报》上只署姓名的第一字母埃 \textbullet 伽 的缘故 . 但发表这封信的时间和整体语调可使人不致怀疑作者的 身分 . 

2)\emph{埃瓦里斯特 \textbullet 伽罗瓦的笔记}  \quad 伽罗瓦去世以后 , 他的全部 文件由奥古斯特 \textbullet 舍瓦烈加以收集 , 并转交给约瑟夫 \textbullet 刘维 . 后 者在他的杂志上只刊登了其中数学方面的著作 . 其余的全部材 料 , 也就是本书现在刊登的 , 曾经当时担任师范大学副校长的 朱利 \textbullet 汤内里在他的著作《埃瓦里斯特-伽罗瓦的手稿》(1908 年巴黎哥底叶 \textbullet 威腊尔[Gauthier-Viüars]出版社版)中加以登 载过 . 不过朱利 \textbullet 汤内里省略了其中的一段 , 即从“会变得 这样容易……”等字起 , 到“起初数学曾具有这种性质”一语 为止 . 

3)\emph{关于被开除出师范大学的文件}  \quad 伽罗瓦致他的同学们的 信和其余的信一起都放在第一部分 , 但《学校公报》上署名为《师范 大学一学生》的简讯则放在此处 . 

4)\emph{埃瓦里斯特  \textbullet 伽罗瓦的诉讼案}

5)\emph{埃瓦里斯特 \textbullet 伽罗瓦的数学著作}

\section{埃瓦里斯特 \textbullet 伽罗瓦的书信}
\begin{center}
	致师范大学同学们

(1830年12月30日星期四《学校公报》)

埃瓦里斯特-伽罗瓦致师范大学同学们
\end{center}

同学们:

在《学校公报》发表了一封简单地署名为《师范大学一学生》 的、谈到我们校长吉尼奥的匿名信 . 你们认为有责任对写信人在 信中叙述的事实所作的解释提出抗议 . 

你们只是在吉尼奥根据单纯的怀疑 , 正如他本人所承认的 , 根 据由来已久的成见把我当作是写这封信的人而开除出校之后才在 这份抗议书上签字的 . 

不论是你们 , 还是我 , 都无法最后决定 , 吉尼奥是否有权 这样做 . 但是 , 将我被开除的责任全部推到你们身上 , 你们是 不应该容忍的 . 在我离校以后 , 你们居然表现出如此这般的兄 弟情感 , 难怪吉尼奥就敢于宣称 , 开除是根据你们的倡议而采取 的了 . 

不错 , 我由于被拒绝提供物质上的帮助而被迫离校之前 , 有人 劝你们采取“正义行动” , 尽管没有什么东西可以拆散我们之间的 团结 , 但有人通过监视人赫伯(Haiber)劝你们反对我将来重返校 门 . 你们拒绝了这种无耻的建议 . 同学们 , 不要停留在这一点上 .  我不为我个人要求任何东西 , 但请你们凭你们的良心和人格所吩 咐的而说话吧 . 你们已经推卸了你们以为是写信人推到你们身上的责任 . 现在就请你们驳斥那种令人更难容忍的说法 , 如果你们 沉默的话 , 那就会变成支持更有力的人的决定了 . 在大臣作出决定之前仍然是你们的同学 , 而且是终身忠实于你们的同志

\begin{flushright}
	埃 \textbullet 伽罗瓦
\end{flushright}

\begin{center}
	1831年1月2日《学校公报》(信前头的标题是:《论 数学教育 . 教员、科学著作和主考人》)
\end{center}

编者先生:

如荷同意登载下述有关巴黎各中学学习数学的看法 , 我将不 胜感激 . 

首先 , 在谈到科学时 , 一位科学家的社会观不应起任何作用; 科学的职务不能成为任何一种政治见解或宗教见解的报酬物 . 使 我感到兴趣的 , 只是教师的好坏 , 至于他对任何问题(科学问题除 外)的看法 , 都与我不相干 . 是不是能够毫无痛苦和愤慨地谈到 ,  在复辟时期只有最起劲宣布自己的帝制派信念和宗教信念的人才 能获得职位?而今情况也没有改变;庸碌之徒依然享有特权 , 尽管 他们对新制度除厌恶外 , 毫无好感 . 话又说回来 , 在谈到科学的功 绩时 , 政治见解就不应考虑在内 . 

我们且从中学谈起 . 大多数学习数学的中学学生 , 都准备投 考综合技术学校;为了帮助他们达到这个目的 , 究竟可以做些什么 事情呢?在叙述最简单的方法时 , 是否有人尽力促使他们去领会 真正的科学精神呢?对他们说来 , 推理的能力是否正在变成第二 种记忆呢?或者适得其反 , 学习数学的方法越来越近于学习法语 和拉丁语的方法呢?从前 , 一位教师可教给学生所需要的一切 .  现在 , 要培养一个埃合技术学校的预备生 , 却需要一个到两个补习 教师 . 

不幸的年轻人要到什么时候才不再整天听讲或死记听到的东 西呢?他们什么时候才有时间去思考他们得到的一大堆资料 , 并 理解许多杂乱无章的一大堆定理以及彼此毫无联系的代数变换呢?要求学生使用最简单的具有一般意义的方法、变换和推理岂不更好?但是不 . 那些废话连篇而又经过阉割的理论却要人们加 以细心研究 , 而最出色的、最普通的代数定理倒被忽略掉了;本来 学生们应该理解这些定理的 , 却反而要去熟悉冗长的、并非永远都是正确的运算 , 并且去论证一些不说自明的结果 . 

坏事的原因何在呢?当然不在中学的教师、他们表现出最值得称赞的热诚 . 他们首先由于数学教学变成单纯手艺而叫苦连 天 . 坏事的根源在于推销主考先生们著作的书商 . 他们需要大部头的书籍;书里各种各样的资料越多 , 就越是有利可图 . 这就是为什么我们可以看到 , 年年出版一些内容浩繁的汇编 , 在这种 汇编中掺杂着年高望重的学者的残缺不全的想法和中小学生的推论 . 

另一方面 , 为什么主考人只向考生提问错综复杂的问题呢? 可能是他们深怕自己被他们的投考人所识破吧;这种在问题里塞 进人为困难的可悲做法从何而来呢?莫非有人认为科学过于简单 了吗?但由此可以得出什么呢?这样一来 , 学生关心的并非获得 教益 , 而是要求考试过关 . 他不得不为每个定理准备四个答案 , 以 应付四个不同的主考人;他必须研究他们惯用的方法 , 并且事先不 仅要背熟对每个主考人的每个问题的答案 , 而且要学会在对答时 如何才能沉得住气 . 因此 , 有充分权利可以说 , 若干年以前 , 已经 出现了一门新科学 , 它一天天地取得越来越重要的意义 . 这门科 学是研究主考老爷们的癖好和他们的情绪 , 研究他们在科学上爱 好什么 , 厌恶什么 . \footnote{关于改组综合技术学校的指示使人可以指望将来的主考人应由科学院提名任 命 , 但不知是每年任命呢 , 还是出现空缺时才予以任命?我们宁愿主考人的职位是临 时性的 , 而且在临考之前才加以任命 . }
你们很走运 , 你们已经幸运地通过了考试 . 此外 , 你们甚至被认为是巴黎人佩服倒地的二百位数学家之一 . 你们以为你们已经 达到目的了吧?你们错了 , 我将在下一封信里向你们证明这一点 . 

\begin{flushright}
	埃 \textbullet 伽罗瓦
\end{flushright}

\begin{center}
	致法国科学院院长

(保存在科学院秘书处的档案)

1831年3月31日
\end{center}

院长先生:

我不揣冒昧 , 希望拉克鲁阿和泊松两位先生不至于因为我提 到三个月前委托他们两位审查的有关方程理论的研究报告 , 而感 到不愉快 . 

这份报告中所阐述的研究结果 , 是去年应征数学优秀作品奖 而提出的著作的一部分 . 我在这份报告中研究了一些规则 , 借助 这些规则 , 在任何情况下 , 一给定方程是否可用根式解出就可以确 定下来 . 因为至今数学家都认为这个问题虽然不是绝对不可能的 ,  但无论如何也是十分困难的 , 所以 , 委员会就预先决定 , 我对这个 问题无能为力 , 第一 , 因为我叫做伽罗瓦 , 第二 , 因为我是个大 学生 . 于是我的研究报告被埋没在委员会里 . 我接到通知说它遗 失了 . 

这对我可以说是够大的教训了 . 但虽然如此 , 我还是遵照科学院的一位荣誉院士的劝告 , 重写了部分原稿 , 并把它呈交 给您 . 

院长先生 , 您看到 , 目前人家对待我的工作 , 正如通常解决方 圆法的问题那样 . 情况是否将要一成不变地下去呢?敬祈院长先生恕我烦渎 , 并请转询拉克鲁阿和泊松两位先生: 我的手稿是否将要再次遗失?抑或彼等准备向科学院报告该手稿?并祈复示 . 院长先生 , 请接受您的忠实仆人对您的诚挚的崇高敬意 . 

\begin{flushright}
	埃 \textbullet 伽罗瓦
\end{flushright}

\begin{center}
	1832年5月25日致奥古斯特 \textbullet 舍瓦烈

(原载1832年《百科全书派评论》9月号)
\end{center}

我亲爱的朋友!

倘能使你获得慰藉 , 忧愁又有何妨;若有好友 , 纵使受苦也能 感到真正的幸福 . 你那封充满圣徒式抚慰的来信 , 使我稍得平静 .  但我经受过的汹涌的激情将如何使之消灭痕迹呢?

既然在一个月之内尽享被释放者的最甜蜜的快乐 , 但他酒醉无欢 , 也无希望 , 你又知道他已永濒山穷水尽之境 , 那又如何能够 感到快慰呢?

啊!于是有人宣扬逆来顺受!于是有人要求受苦人对世界慈悲为怀 . 慈悲么?决不!憎恨 , 只有憎恨!谁对目前不感到刻骨的憎恨 , 而对未来不怀着真挚的热爱呢?

如果我的理智不再需要暴力 , 那么我的内心却需要暴力 . 我 要为我经历过的许多苦难复仇雪恨 . 

要是没有这一切 , 我就可以和你在一起 . 

不过 , 我们且撇开这些吧;有一种人 , 命运决定他们来做好事 , 而永远不坐享其成 . 我担忧 , 我是属于这种人的 . 

你说 , 那些喜欢我的人应当帮助我解决日常生活的困难 . 可是你知道 , 喜欢我的并不很多 . 而对你说来 , 帮助我就意味着竭尽一切可能来满足我的愿望 . 我认为自己有责任——我已这样做过成百次一警告你 , 你这种努力都是徒劳无益的 . 

我仍然怀疑你的悲观预言:我不再从事科学工作了——这句 话的真实性 . 不过我承认 , 这种预言并非毫无根据 . 妨碍我成为 科学家的 , 恰好是我不光是个科学家 . 我内心激愤得违反理智 , 怛我不象你那样补充说:“非常遗憾 . ”

可怜的奥古斯特 , 如果我触伤了你的情感 , 并轻率地批评了你 所信赖的人 , \footnote{指昂方坦 . ——俄译者}那就请原谅吧 . 向他瞄射的箭矢 , 并不太尖锐 , 而在我的笑声中并没有苦味 . 对我的忿恨心情说来 , 这已经够了 . 

我将在6月1日来拜访你 . 我希望 , 6月上旬我们会经常见面 . 15日我将到多飞内(Dauphiné)去 . 

\begin{flushright}
	一切都是你的

埃 \textbullet 伽罗瓦
\end{flushright}

附:我重读你的来信时 , 注意到你有一句话 , 责备我陶醉在曾 经玷污我的心、头脑和双手的腐烂世界的腐臭气息之中 . 

这种激烈的责备恐怕在暴力制度拥护者那里是找不到的 . 

陶醉!我对一切 , 甚至对荣誉的爱好也都感到失望了 . 我所 憎恨的世界怎么还会玷污我呢?请好好地想一想吧 . 

\begin{flushright}
	1832年5月29日致奥古斯特 \textbullet 舍瓦烈

(原载1832年《百科全书派评论》9月号) 
\end{flushright}

我的亲爱的朋友!

我在分析方面有了某种新发现 . 其中有些涉及方程论 , 另外一些涉及整函数 . 

在方程论中 , 我探讨了在哪些情况下可以用根式解方程 , 这给我一个理由来加深这个理论 , 并描述各种各样的甚至用根式都不能求解的方程变换 . 

从这里可作出三份研究报告 . 第一份已经写成 , 而且经过修改后 , 不管泊松说了些什么话 , 我坚信它是正确的\footnote{这是一封唯一只摘引片断的信 , 除了这里登载的开头和结尾以外 , 它完全都是 谈数学的 . } . 

…………

我亲爱的奥古斯特 , 你知道我不是单单研究这些问题而已 .  从某个时期起 , 我考虑得最多的 , 是怎样把不定式理论应用到超越分析上面 . 问题是要先验地预见到在超越数或超越函数的关系中可以进行什么样的变换 , 即哪些数可以用来代替该数 , 使关系仍然 有效 . 这就使人不得不承认有很多公式是不可能成立的 , 它们必 须用另一种方法来进行研究 . 可是我没有时间 , 而我对这个辽阔 领域的认识还是不很清楚 . 

让我这封信刊载在《百科全书派评论》上吧 . 

我一生中不只一次地敢于提出我没有把握的命题 . 这里所写 的一切 , 我已经考虑了将近一年了 , 对我自己的需要讲来 , 我是不 可能弄错的 , 否则人家就会怀疑我只指出一些定理 , 而对于它们的 充分论据却茫然无知了 . 

请你公开向雅科比和高斯请教 , 并请他们发表自己的意见 , 但 不是谈论定理正确与否 , 而是谈论这些定理的意义 . 

我希望 , 这样一来 , 将会出现一些人 , 他们会认为 , 把这种杂乱 无章的情况整理出秩序来 , 对于他们自己是大有裨益的 . 

热烈地拥抱你

\begin{flushright}
	埃 \textbullet 伽罗瓦
\end{flushright}

\begin{center}
	1832年5月29日致全体共和派

(原载1832年《百科全书派评论》9月号)
\end{center}

致全体共和派的信

我请求我的爱国朋友们不要责备我不是为自己的祖国而献出生命 . 

我将成为一个下流的卖俏女人的牺牲品而死去 . 我的生命要 在一场下流的丑事中消灭掉 . 


啊!为什么要为这样无聊的事情而死去 , 为什么要为这样可 鄙的事情而死去!
苍天作证 , 我曾用尽办法试图拒绝这场决斗 , 只是出于迫不得 已才接受了挑战 . 我后悔对这些不善于冷静地倾听真情的人说了 招来不幸的实话 . 但是 , 我终归是说了真话 . 我将带着不受谎言 ,  不受爱国者的血所沾染的良心到坟墓里去 . 

永别了!我已经为公共的幸福献出了自己大部分的生命 . 

不要责备杀死我的人 . 他们是诚挚的……

\begin{flushright}
	埃 \textbullet 伽罗瓦
\end{flushright}

\begin{flushright}
	1832年5月29日致N .  L . 和V .  D . \footnote{在信纸背面伽罗瓦写下四个名字:V . 德郎诺伊(V .  Delannoy) ,  N . 勒本(N .  Lebon) , F . 盖韦斯(F . Gervais)和A . 舍瓦烈 . 很可能信是写给勒本和德郎诺伊的(他们 是谁 , 迄今无人知晓)【直到现在 , 通常认为姓名第一字母V . D . 系影射杜沙特列 . 但是这种假设跟本 书提出杜沙特列是伽罗瓦决斗的对手的说法是无法并存的 . ——俄译本编者注】}

(原载1832年《百科全书派评论》9月号)
\end{flushright}

致N . L . 和V . D . 的信

亲爱的朋友们!

有两个爱国者约我决斗……我无法拒绝 . 

请原谅我没有通知你们之中的任何一个人 . 

我的对手们要我提出保证不预先通知任何一位爱国者 . 

你们的任务很简单:你们应当证实 , 我是违背自己的意愿而参 加决斗的 , 也就是说 , 用尽一切办法希望和平调解事情之后 , 才进 行决斗的;你们还应当证实 , 我对无聊的事 , 甚至象上述的无聊事 ,  也不善于撒谎 . 

不要忘了我!因为命运不让我活到祖国知道我的名字的 时候 . 

我至死还是你们的朋友 . 

\begin{flushright}
	埃 \textbullet 伽罗瓦
\end{flushright}

\section{埃瓦里斯特 \textbullet 伽罗瓦的笔记}
\begin{center}
	(科学院图书馆 , 伽罗瓦全档 , 第9卷宗 , 第59 — 61页)
\end{center}

\begin{center}
	\emph{初步意见}
\end{center}

这份研究报告大约在七个月以前寄给法国科学院 , 受委托审查它的人把它遗失了 . 可见 , 这份报告殊少趣味以致不堪卒读 .  在阻挠作者发表该报告的人们当中 , 这种想法并非最新的想法 .  倘若作者仍然决定发表 , 只是由于担心另有从事同一研究的较机 敏的数学家会攫取他的长期劳动的成果而已 . 

报告所追求的目的是寻找用根式解方程的特征 . 我敢断言 ,  在纯粹分析中 , 没有比这个问题更加暧昧不明 , 而又跟其余一切问 题更加无关的了 . 新颖的问题需要使用新名称、新符号 . 我不怀 疑 , 这种不方便在开头的时候将使读者产生反感 , 他们很难原谅这 种生疏的语言 , 即使作者是他们素所景仰的人 . 但是归根到底 , 我 们只好适应题目的要求 , 因为题目的重要性值得注意 . 

我们这里准备全面答复的问题是 , 在一个给定的具有任意数 字系数或字母系数的代数方程中 , 它的根是否能够在根式中表示 出来 . 

如果你们向我提出由你们随便选择的一个方程 , 并且想知道 该方程是否可以用根式解出 , 那末我所能做到的唯一的事情 , 就是 告诉你们怎样才能解答这个问题 , 但不是打算让谁来实行我的指 示 . 简言之 , 所需要的计算实际上是做不到的 . 


这样 , 就可以看出 , 从我们所提出的解答中 , 人们不能得到任 何好处 . 

如果问题的提法都象上面那样 , 那么 , 结果就会是这样的 . 但 在大多数代数分析的应用中 , 我们得到的是那些属性业已为人所 知的方程 . 如果我们往下准备叙述的这些属性和规则是已知的 ,  那我们总是能够轻易地来回答问题 . 事实上 , 对于这些类型的方 程说来 , 总是存在着一连串抽象的推论 , 它可以决定所有的计算并 使这些计算变成多余的东西 . 我就举一些方程为例 , 这些方程可 以决定椭圆函数 , 而且已经由著名的阿贝尔解出 . 这位数学家之 得出上述方程完全不是根据方程的数字形式 . 这一理论的美妙和 繁难之处就在于必须不断地给计算过程作出指示(尽管实际上并 不进行这种计算) , 并能预见其结果 . 我以后还要用模方程为证 . 

\begin{flushright}
	(1830年9月写毕)
\end{flushright}

\begin{center}
	(科学院图书馆 , 伽罗瓦全档 ,  第12卷宗 , 第74 — 78页)
\end{center}

\begin{center}
	\emph{数学科学论纯粹分析的进展}
\end{center}

众所周知 , 在人类的全部知识中 , 纯粹分析是一个最非物质 化、最合乎逻辑、而又是唯一地完全不以感性认识为转移的知识部 门 . 因此很多人得出结论:纯粹分析的各个部分彼此极其一致 , 而 在方法上又是同格的 . 但这是一种错误的看法 . 随便拿一本代数 书一不论是教科书还是科学著作 , 其中除了大量乱七八糟的定 理(它们的严整性跟普遍存在的混乱形成了奇特的对照)外 , 你们 什么也找不到 . 看来 , 有些观念使作者大费周章 , 以致他无力把它 们贯串起来 , 而他的头脑已被作为著作基础的各种概念弄得精疲 力竭 , 再也不能产生一个把这些概念连结在一起的思想了 . 

如果你们碰到任何一种方法、任何一种联系或者任何一种的 对比 , 那它们必然是错误的和伪造的 . 章节划分得全无根据 , 层次 是任意的 , 编排方法也是因循旧习的 . 这种比缺乏方法更为严重 的缺点 , 在教科书中尤其常见 , 因为在大多数情况下教科书的作者 并没有充分认清他们所写的是什么东西 . 

没有跟科学打过交道的人 , 必定会感到这一切都很奇怪 , 因为 他们通常把“数学”一词看作是“精确”一词的同义语 . 

但如果他们考虑到 , 科学不过是人类智慧的这样一种产物 , 它 注定了去研究和探索真理 , 而不是发现和认识真理 , 那他们就不至 于这样地大惊小怪了 . 事实上 , 如果有足够深邃的智慧 , 能够立即 掌握到不仅是我们已知的东西 , 而且是各种各样一般的数学真理 的全部总和 , 那么就有可能借助某些普通原理中的同样方法 , 合乎 逻辑地、又似乎机械地引申出这些真理;这样 , 就不会再有科学家 在研究工作中所常碰到的障碍和困难了 . 但是 , 这也不尽然 . 科 学家的任务更为艰巨 , 因而也就更为美好 , 科学的发展又是比较 不平衡的 , 因为它要通过一系列的配合才能得到发展 , 而在配合之 中 , 偶然性所起的作用远非微不足道的;科学的生命是混沌一团 的 , 它好比由于矿层的毗连而相互交错的矿物一样 . 这种情况不 仅适用于由众多科学家之工作成果所构成的整个科学界 , 而且也 适用于其中每一个科学家的单独研究工作 . 分析家们用不着欺骗 自己 , 因为他们并不是在演绎真理 , 而是在进行组合;他们之领会 真理 , 也是徘徊于左右之间 . 

每当涉及从来未曾研究明白的题目时 , 不论是教科书或是科 学著作 , 都犯有同样毛病——叙述欠明确 . 只有范围十分狭小的 问题才能在方法上得到应有的透彻的研究 . 可是要做到这一点 ,  就需要高深的分析知识 , 但这整个工作做来无益 , 因而又使能够胜 任这项任务的人弃而不顾 . 

跟类似这种的缺点作斗争 , 不先克服自己对某一科学家的个人好恶是不严肃的 . 本文作者将避免这类暗礁 . 如果痛苦的往事保证他具有前一种感情 , 那么对科学的深刻的爱好 , 将使他尊敬为科学服务的人 , 也将保证他不偏不倚 , 并使后一种感情成为不可能 . 

在科学中 , 很难只以批评的任务为限 , 我们之所以这样做 , 只 是迫不得已的 . 在进行了批评之后 , 只要我们力之所及 , 我们将指 出按照我们的意见认为是正确的东西 . 这样一来 , 我们才有机会提醒读者注意与研究分析有关的新概念 . 我们在头几篇论文里就使读者取得这些概念 , 旨在以后不再回到这些问题上面来 . 

如果涉及的是比较不那么抽象的东西 , 譬如 , 艺术的对象 , 却把批评别人文章放在自己的作品前面 , 那就令人可笑了 . 这就意味着过分天真地去一再申述在实质上几乎永远是真实的事情 , 这 也就是说 , 过分天真地去一再申述自己是一个有资格对于与自己 有关的事情作出判断的人 . 但是 , 在现在的情况下 , 涉及的并不是 具体的成就 , 而是人类所能理解的一切概念中的最最抽象的概念; 在这种情况下 , 批评与讨论是同义语 , 而讨论就是把自己的思想和 别人的思想进行对比 . 

总之 , 我打算在若干篇论文中阐述由于成千上万种的情况以 致过去未能发表出来的学术著作中的最普通、最玄妙的部分 . 我仅仅提出这些论点 , 不用那些通常会使分析家们完全迷糊的关于一般概念的例证和补充而把这些论点复杂化起来 . 要想把这些论点叙述得极其认真 , 我将实事求是地讲述导致我得出这些论点的 途径 , 讲述我必须加以克服的一些障碍 . 我希望读者象我一样地 弄清楚这些问题 . 一旦达到这个目的 , 我就可以认为自己做了一 件好事 , 虽然不是因为我给科学界直接带来好处 , 究竟是因为我树 立了迄今未曾有过的认真的榜样 . 

\begin{flushright}
	( 1832年3月29日至4月5日之间写于圣佩拉吉监狱 . )
\end{flushright}

\begin{center}
	(科学院图书馆 , 伽罗瓦的全档 ,  第13卷宗 , 第79 — 80页)
\end{center}

在这里 , 正如在所有科学中一样 , 每个时代都提出自己当时存 在的若干问题:il y a des questions variantes qui fixent à la fois les espri ts les plus éclairés comme malqré eux et sans que [illis .  ]ait présidé à ce concours . \footnote{此句由于原手稿方括弧处字迹不清 , 文意不清楚 , 无法译出 . ——译者}看起来 , 往往是许多人同时产 生同一的思想 , 仿佛是某种启示使他们同时恍然大悟似的 . 如果 要找出这个原因 , 只要浏览我们前人的著作 , 就很容易把它发现出 来;那些同样的概念早已存在在前人的著作里 , 虽然目前还不知道 它们的创始人是谁 . 

直到目前为止 , 科学还没有从科学家的学术研究所经常出现 的那种巧合中得到很大的利益 . 见不得人的竞争 , 卑鄙的勾心斗 角——这就是他们唯一的成果 . 但是 , 在这个事实上 , 不难发现这 样证据:科学家生来并不比所有其余的人要过孤独的生活的;他们 也是属于特定时代的人 , 而且迟早要协同合作 . 到了那时候 , 将有 多少时间腾出来用于科学!现在 , 分析家们对许多完全新型的问题发生兴趣 . 我们的任 务即在于揭示(这些问题之间的联系)\footnote{此句括弧内的字系后人添加的 . ——译者} . 

\begin{center}
	(科学院图书馆 , 伽罗瓦的全档 ,  第11卷宗 , 第67—73页)
\end{center}

\begin{center}
	有关埃 \textbullet 伽罗瓦的分析的两篇研究报告
\end{center}

\begin{center}
	\emph{序}
\end{center}

\begin{flushright}
	这是一本诚实的书\ , 蒙台涅\footnote{M .  de Montaiqne , 1533-1592 , 文艺复兴时期法国著名哲学家 , 《经验论》作者 . ——译者}
\end{flushright}

首先 , 翻开这部著作 , 人们看不到一大堆对于遇人奉承则稍稍 解囊 , 而恭维不够则一毛不拔的吝啬的显贵们的颂词和接连下来 的一大堆姓氏、名字和称号 . 在这里 , 谁也看不到用斗大的字写成 的用来对科学巨子或对某某学术保护人表示景仰的标题;这样的 学术保护人对于一个在二十岁就想写作的人说来是十分有益(我 甚至要说 , 十分必要)的 . 

我不谈我的著作中的所有的优点乃是深受某人的劝告和鼓励所致 . 我之所以不谈 , 是因为这样谈就是意味着撒谎 . 如果我要对当今世界的伟人 , 或者科学界的伟人(两者的区别现在已经觉察不出来了)说些什么话 , 那么我发誓 , 这并非什么感激的话 . 我的两 份研究报告发表得这么晚 , 应当归咎于科学界的伟人;至于这两份都是在监狱里 , 在一个未必适于集中精力进行工作的地方写成的:

在那里 , 我经常因为堵塞愚蠢的左伊尔(Zoile)\footnote{左伊尔(Zoilus , 约纪元前第三世纪) , 希腊历史学家和批评家 . 他激烈攻击荷马 , 曾被称作“荷马的鞭笞” , 第欧尼修斯把他看作卓越的批评家之一 , 但是亚里士多德对他持着敌视态度 . 由于后者 , 左伊尔的名字后来成为心怀恶意的和吹毛求疵的批评家的代名词 . ——译者}的嘴巴的任务很少光顾到我的头上而感到惊讶——这却要归功于当今世界的伟人了 . 我的反对者在我的心目中是如此之低劣 , 以致看来我有权使用左伊尔的语言 , 而不至于使自己的谦虚受到怀疑 . 我不打算讲述我是为了什么和因为什么而入狱 , 但我必须讲述手稿是如何经常地在科学院院士老爷们的文件夹里遗失掉 , 尽管我对那些对于阿贝尔之死而问心有愧的人的那种相似的疏忽大意 , 是完全无法理解的 . 我根本不想把自己跟这位大名鼎鼎的数学家相比;我的研究方程论的报告确实在1830年4月间提交科学院 , 研究报告的摘录也在1829年寄给科学院 , 我没有接到任何答复 , 甚至要重新看到手稿也简直成为不可能——只须提到这一点就够了 . 在这方面有一些很有趣的奇闻 , 不过我不想加以叙述 , 因为对我个人说来 , 除了手稿的毁灭以外 , 类似这种的事情不会再发生了 . 一个旅客竟因身体瘦削而得虎口逃生 , 真是万幸 , 我说得过多 , 已足使读者明白 , 为什么不管我的意图如何 , 我决不能用献词来美化 , 或者 ,  如果愿意的话 , 来丑化我的著作 . 

其次 , 和标题比较起来 , 两份研究报告实在少得不相称;报告中有多少代数公式 , 至少也有同样多的文字 , 结果 , 当手稿交给排字工人时 , 他天真地断定这是引言 . 在这一点上 , 我是不应该得 到任何宽恕的;当然借口有必要把手稿写成通俗易懂的形式 , 并在 它的基础上编出完整的理论 , 那是轻而易举的 . 或者 , 不如毫不客 气地用两三条完全无关的新定理插入某一科学部门 , 岂不更妙!把字母表的全部字母依次代入每一方程 , 并把所得的全部比值都编上号码 , 以便有可能区别紧接在后边的方程是属于哪一种字母的组合 , 也是轻而易举的 . 这会使方程的数字增加到无穷大 , 尤其是如果记住除了拉丁字母以外 , 还有希腊字母 , 如果希腊字母用完了 , 还有德国字母;甚至使用阿拉伯字母 , 必要的话 , 就是使用中国的象形文字也是毫无障碍的!把每个词修改十次 , 而念念不忘地要在每一种异文之前安上一则郑重其事的标题:《定理》 , 并借助我们的分析求得十种自从光荣的欧几里德时代以来就已闻名的结 果 , 也是轻而易举的事情;或者 , 最后 , 在每一条定理之前和之后 ,  插进一连串有疑问的专门例子!可能性有的是 , 可是我没有能够利用其中的任何的一个!

第三 , 第一份研究报告已经提交内行人过目 . 1831年寄往科学院的这份报告的摘录 , 已交泊松先生审阅;他说 , 他一点也看不懂这份报告 , 对我这个被著作者的自尊心弄得眼花缭乱的人说来 , 这除了意味着泊松先生不想理解 , 或者无法理解之外 , 别无其他意 义;但对大众说来 , 这却令人信服地证明 , 我这本书是分文不值的 . 

这样一来 , 所有一切都使我认为 , 我向大众提出的著作 , 科学界将带着怜悯的笑容予以接受 , 甚至最宽宏大量的人也会指责我的无能 . 在若干时间内 , 我将被比拟为符伦斯基\footnote{约 \textbullet 符伦斯基(J . Wronski , 1778-1853) , 波兰数学家 , 由于引进函数行列式而 闻名 , 但是 , 他对自己的研究阐述得极其含混 , 以致当时的大多数数学家都表示怀 疑 . ——译者} , 或者比拟为那些每年都找到方圆问题新解答的孜孜不倦的勤学者 . 我的著作引起综合技术学校主考先生们不可抑制的哄堂大笑 . (顺便说一句 , 我很奇怪 , 他们在科学院一再地占不上一个位置 . 在后人的记忆 中 , 也未必会找到他们的位置的 . )正是这些竭力把所有数学书的发行一把抓在手里的先生们 , 自己会突然发现 , 曾经两次遭到他们批驳的某某年轻人 , 竟然胆敢写起书来 , 而且写的还不是什么教科书 , 却是独立的科学著作 . 

……

我写出上述的一切 , 是为了证明我是故意让自己成为蠢人们 取笑的对象的 . 

起初 , 数学具有这样一种性质:即代数方法的计算已不十分 需要;非常简单的定理未必值得翻译成分析的语言 . 只有在欧拉 以后 , 由于这位伟大的数学家为科学界发现了新的可能性 , 这种 较为简短的语言才成为必不可少的东西 . 从欧拉开始 , 计算变得 越来越需要 , 同时 , 随着它之应用于越来越高深的科学部门而变 得越来越繁难了 . 本世纪初 , 算法已经达到如此复杂的程度 , 以致 现代的数学家们若不使自己的学术著作具备严整性 , 能够迅速地、一目了然地掌握大量的运算数目 , 则任何进步都是不可能的 . 

显而易见 , 如此受人赞扬 , 如此必不可少的科学严整性 , 不可 能有别的目的 . 

从最先进的数学家们致力于求得严整性这一确定不移的事实 中 , 可以有把握地作出结论说 , 同时掌握几种运算的必要性变得越 来越迫切了 , 因为人的智力没有足够的时间来详细研究细节 . 

总之 , 我以为 , 依靠改进计算而获得的简化(这里当然是指原 则上的简化 , 而不是指技术上的简化)绝对不是无穷无尽的 . 终有 一天 , 数学家必将能够如此清楚地对代数变换作出预见 , 以致不必 再花时间和纸张来认真进行计算 . 我并非断言 , 除了这种预见外 ,  分析不能有其他的新成就 , 但我认为 , 如果没有这种辅助工具 , 有 朝一日 , 全部分析方法都将成为徒劳无益的东西 . 

使计算听命于自己的意志 , 把数学运算归类 , 学会按照难易程 度 , 而不是按照它们的外部特征加以分类——这就是我所理解的 未来数学家的任务 , 也是我所要走的道路 . 

但愿任何人都不要把我流露出来的急躁情绪跟某些数学家向 来对无论哪一种计算都要根本回避的意图混为一谈 . 他们不用代 数公式 , 而使用冗长的议论 , 在重迭的数学变换之上 , 又加上对这 些变换的重迭文字的概述 , 所使用的又是不适于解答这些算题的 语言 . 这些数学家落后了一百年 . 

在这里没有类似的情况 . 我在这里进行分析之分析 . 与此同 时 , 现在已知的变换中最复杂的变换(椭圆函数) , 只不过被看作十 分有益的甚至必不可少的、但究竟是个别的情况 , 因此拒绝作进一 步的、更广泛的探讨 , 将是一种不可挽救的错误 . 总有一天 , 在这 个粗具轮廓的高等分析中所提到的变换 , 将真正地得到实现 , 并且 是按照难易的程度 , 而不是按照这里出现的函数形式予以分类 . 

我所提出的一般论题 , 只有在仔细阅读我运用这个论题于其中的著作后才能理解 . 理论原理不应先于实际运用 . 写完这本书以后 , 我反躬自问:为什么大多数读者觉得这本书很奇怪 . 在考察自己的思想时 , 我认为 , 原因在于分析我所谈论的任何问题时 , 我 希望回避计算;不仅如此 , 我得承认 , 想进行这些计算的人 , 在大多 数情况下遇到了不可克服的困难 . 

必须加以注意的是 , 在考察这样新颖的问题 , 走着这样不平凡 的道路时 , 我经常碰到一些我无法克服的困难 . 因此在两篇研究报告中 , 特别是第二篇 , 经常可以遇到这样的句子:“我不知道 . ”我开头提到的那些读者 , 是不会放过嘲笑我的机会的 . 不幸的是 , 作者 对读者隐瞒困难 , 会带来多大的害处 , 那是难于设想的 . 当竞争 , 也 就是利己主义在科学界不再时行 , 当科学家们不再把印刷包裹寄 到科学院去 , 而是协同工作的时候 , 每个人将急于发表最微不足道的资料 , 只是因为资料很新颖 , 对于其他的资料就说:“我不知道 . ” 

\begin{flushright}
	1831年9月寄自圣佩拉吉监狱

埃瓦里斯特 \textbullet 伽罗瓦
\end{flushright}

\begin{center}
	(科学院图书馆 , 伽罗瓦的全档 , 第14卷宗 , 第81页)
\end{center}

\begin{flushright}
	一切都要看看 , 都要听听 , 但唯一的思想不要丢掉 . 

《科学报》(Science) 1831年9月29日
\end{flushright}

\begin{center}
	\emph{等级制度——学校\footnote{这是一份笔记 , 含义隐晦 , 标题中“学校”二字也是隐喻之意 . ——译者}}
\end{center}

等级制度甚至对地位卑下者说来也是一种工具 . 

任何没有嫉妒的人或是胸怀大志的人都需要人为的等级制 度 , 以便克服嫉妒和障碍 . 

目前人们还不能说:“科学——这就是我” , 他应当有个名字 ,  才好把他跟他与之作斗争的事物进行对比 , 否则他的功名心将被视为嫉妒 . 

是王者 , 必先是贵族 . 马基雅弗利\footnote{Machiaveli , 1469-1527 , 文艺复兴时期意大利政治家和著作家 . ——译者}的话 . 

阴谋是游戏 . 如果你试图去做的事情成功了 , 你就会赢得一 切 , 如果不成功 , 只会失去一部分 . 

教授们受科学院打击 , 科学院受过去打击 , 过去受另一个过去 打击 . 

例如[此处字迹不清]维克多 \textbullet 雨果\footnote{Vicfor Hugo ,  1802—1885 , 法国浪漫主义派文学的先驱者 . ——译者} . 文艺复兴 , 中世纪、以 及我 . 

由于一个人需要战胜另一个人 , 一个时代需要战胜另一个时 代 , 所以产生了科学和文学上的一些持续不断的反动——亚里士 多德、托勒玫(Piolémée)笛卡尔、拉卜拉斯 . 

[这一行字迹不清]

玩这种游戏使人竭精殚虑 . 一个对任何事情都不热心的人会变成折衷主义者 . 

一个有思想的人 , 可以选取学者生前的显赫声望 , 或者选取创 办学校、沉默以及未来的大名气 . 在前一种情况下 , 他会实现自己的思想 , 而无须让人知道 , 在后一种情况下 , 他宣扬他的思想 , 做 到无人不知 . 还有一种介于两者之间的中庸之道:多方宣扬并加 以实行 . 这样一来 , 他就是荒谬可笑的了 . 

\section{关于被开除出师范大学的文件}
\begin{center}
	《学校公报》摘录 . ( 1830年12月5日)
\end{center}

\begin{center}
	\emph{答复吉尼奥先生和《中学报》}
\end{center}

我们把吉尼奥先生称之为“温文尔雅的师范大学校长” , 显然是刺痛了他 . 他想要保持道学先生的语言、腔调和各种风度 , 想看 到自己家里有“沙龙” , 并且要我们对此只字不提 . 我们认为这就 是一个警告 . 好吧 , 我们再不应对他的名字进行屡受社会良心谴 责的粗暴的私人攻击了 . 我们非常了解 , 对于竭力想远避旁人耳目 而悄悄地走着自己道路的人 , 任何涉及他们个人的意见 , 看来总是 过分粗暴的 . 下一次如果看见一个人左右开弓 , 击败自己道路上的 一切阻碍物时 , 我们就要装出什么也没有看见的样子 , 并拿起长柄 眼镜 , 表示我们的视力很差 . 我们不会毫不客气地责备吉尼奥先生巧妙地利用吉本(Gibbon)先生生病而溜上预科学校教务处主任的 职位 , 我们要对钻营家和阴谋家说几句含糊的话 . 我们不想说 , 吉 尼奥先生刚刚开始主持教务处 , 就开始钻营起来 , 最后他谋到校长 的职务 , 后来又谋到教育总监的职务;我们也不想断言 , 不过仍然 是假定着说 , 他很乐意挑起师范大学总顾问头衔项下的重担 , 只要 他能侥幸谋到这个头衔;我们不会伤心地惋惜师范大学一切都恢 复了老一套 . 我们不管这些 , 宁可一般地去指摘事物过分迅速地 向前进展 , 这样才一定不会给自己、给别人惹来麻烦 . 我们不想叙 述吉尼奥先生如何不满足于这些 , 而尽力设法损害别人来保障自己的生活方面的小福利 , 以及他又如何逼迫一位教授搬出普列两(Plessis)学院的免费宿舍——不管这些 , 我们将保持沉默 , 因为沉 默较有礼貌 . 

至于《中学报》 , 我们只想谈一点 . 在惠赐给我们的十五行的 答复中谈到所有大学教授的职务必须通过竞争而取得一节 , 我们 却找不到对不合理情况的解释 . 但愿报纸在辩论中稍能保持诚 实;也许这会有助于我们更好地相互理解 . 

我们刚刚收到下面的一封信 , 未必有什么比它能更好地补充我们的答复了 . 

\begin{flushright}
	1830年12月3日
\end{flushright}


阁下:

吉尼奥先生昨天在《中学报》上发表的有关贵报刊登一篇文章 的信 , 在我看来是完全不能容忍的 . 我想 , 您一定准备使用一切手 段来揭露这个人 . 

下面就是四十六位学生可以加以证实的事实 . 

7月28日早晨 , 当师范大学的几位学生表示希望参加斗争以 后 , 吉尼奥先生曾两度扬言 , 他准备召唤警察来恢复学校的秩序 .  7月28日的警察!

在同一天 , 吉尼奥先生以他习惯的书呆气告诉我们说:“不少 勇敢的人被左右两派杀死 . 假如我是一个军人 , 我就不能决定做 什么 . 应该牺牲什么呢?是自由还是法律秩序?”

就是这个人在第二天用三色帽徽装饰自己的帽子!这就是我 们自由主义的道学先生!

阁下 , 还应知道 , 充满高尚的爱国主义情感的师范大学学生 们 , 在不算很久以前 , 会见了吉尼奥先生 , 通知他关于他们打算向 国民教育大臣请愿的意图 . 他们希望拥有武器并进行军事训练 ,  以便遇到必要时能够保卫自己的国土 . 且看吉尼奥先生的答复 .  他的为人也象他7月28日的答复一样地是自由主义的:“答应向 我提出来的请求会使我们变成人家取笑的对象 . 这是仿效各院校 所发生过的事情 . 这是不体面的事情 . 我提醒你们注意 , 各院校 向大臣提出这样的要求 , 只有两位皇家会议的委员投票赞成 , 恰恰 这两位并不是自由主义者 . 大臣之所以同意 , 只是因为他害怕学 生们的不安情绪 , 也就是有完全破坏大学甚至综合技术学校危险 的值得令人遗憾的情绪 . ”

不过 , 我认为 , 吉尼奥先生在某一方面有充分理由否认他对新 的师范大学持有偏袒态度的指摘 . 就他而言 , 没有任何东西能比旧的师范大学更美好的了 , 旧的师范大学——就是一切 . 不久以前我们向他申请制服 , 他拒绝了 , 因为旧的师范大学没有制服 . 在旧的师范大学里 , 学习期限为三年;在设立预科学校以后 , 第三学 年被认为是无用的;吉尼奥先生设法恢复了旧学制 . 

象在旧的师范大学里那样 , 我们不久就要每月只限外出一次 ,  并在下午五点钟以前返校 . 遵守库申先生和吉尼奥先生所建立起 来的制度应该是多么惬意!

在我们校长身上 , 一切都表明他的思想极其贫乏 , 墨守成规已 经达到了不可救药的地步 . 

我希望 , 这些详细情形不会使您感到不愉快 , 贵报会从中吸取 应有的好处的 . 

\begin{flushright}
	师范大学一学生
\end{flushright}

编者按:我们发表了这封没有署名的信 ,尽管没有人要求我们这样做;但要提醍人们注意 , 在三天难忘的七月曰子以后 , 所有的报纸都刊登了吉尼奥先生的声明 , 在这篇声明里 , 师范大学校长 向他的学生们转达了临时政府的命令 . 

\begin{center}
	1830年12月12日《学校公报》摘录(副题是:
《师范大学学生们的抗议书》) 

致《学校公报》编者

1830年12月10日 , 巴黎 . 
\end{center}

阁下:

并不是我们这些早就来到师范大学的学生们 , 应当对来自校 外的对我们校长的攻击产生反应 . 因此我们既不愿意与《学校公 报》进行长期争论 , 也不愿意反驳您的毁谤;但是我们深为愤慨的 是 , 你们之中有一个人自命为整个师范大学的代表 , 并以我们的名 义一口咬定他所想象的虚假的、完全颠倒黑白的事实 . 对12月5日《学校公报》上所刊登的信 , 不论在内容上或形式上 , 我们都彻 底地予以否定 . 我们无意赞同该信所表示的感情 , 我们趁此机会 向吉尼奥先生表示我们的谢忱 , 因为在他的任期中 , 每当师范大学 处于紧耍关头时 , 他都以高尚的精神和坚定的态度来保护我们的利益 . 我们声明 , 由于吉尼奥先生 , 师范大学的学生才享有思想自 由 , 而在别的地方 , 思想自由则到处受到摧残 , 同时 , 在7月的最后 几天中 , 他对待我们一如往昔 . 我们呼吁师范大学的本届毕业生 都来作证人 , 我们深信 , 他们将毅然表示要保卫自己以前的校 长 . 阁下 , 我们希望 , 您既然那样急于接受我们之中的一个人的 指控 , 您也会接受所有其他人的抗议 , 而我们的这封信也将会 在贵报最近一期上发表 , 因而我们就无须采取其他正式的抗议手 段了 . 

二年级留校学习并目睹事件发生经过的学生:E . 阿米尔(E .  Hamel)、赫拉尔(Guérard)、J . 杜普勒(J .  Duprey)、卞-拉菲(Bens-Lafaist)、鲁(Roux)、蒙连(Monin)、惠更连(Huguenin) ,  Edw . 巴 里(Edw .  Baryx达巴(Dabas)、A . 卡彼尔(A . Capelie) ,  F . 柯列(F . Collet)F . -V . 范第(F . -V . Vendeyes)、狄马鲁(Desmaroux) . 

再者 , 二年级不曾目睹事实的学生也宣布 , 他们拒绝提供《学 校公报》12月5日所刊登的来信作者要求他们提供的证据 . 

C . 波列(C .  Pollet)、拉萨森(Lassassaigne)、俾谢(Bissey)、皮诺 (Piiiaud)、洛兰(Laurent)、索飞(Choffet)、热拉尔(Gérard)  . 

\par

在同一期的《学校公报》上 , 编者刊登了下列消息:

我们刚刚获悉:师范大学校长不久以前实行某种刑罚 . 他召 集全体学生 , 对每个人进行个别询问 . 他问:“你是《学校公报》上 刊登的那封信的作者吧?”头四个学生矢口否认;第五个听到这个 问题的学生回答说:“校长先生 , 我认为没有权利回答您的问题 , 因为这样就意味着参与告密我们的一位同学 . ”这一充满高尚精神的 坚定态度的答复 , 使吉尼奥先生大为不快 . 

\begin{center}
	\emph{吉尼奥就开除伽罗瓦一事写给大臣的信}
\end{center}

大臣阁下:

我深感遗憾 , 不得不立即向您报告我迫不得已而擅自决定采 取的一项措施 , 并请求您立即予以批准 . 

由于学生伽罗瓦的行为 , 我刚刚予以开除出校的处分 , 并把他 遣返他母亲的家;我荣幸地在本月三日的信里就他的行为奉告过 您 . 正如他的许多同学的声明所表明的 , 在他妄图对学监于梅尔 (Jumel)和我进行抵赖以后所作的横蛮无礼的自白中 , 我坚信 , 从 上星期日以来他的所作所为 , 已引起全校的愤慨 . 这里所说的是 指当天发表在《学校公报》上的那封信——我认为有必要报告这家 报纸的名称——而这封信十分露骨地署名为:“师范大学一学生” .  凡是读过这封信并同我谈过这件事的人 , 都认为它非常严重地损 害了学校的荣誉 , 因此我对它不能熟视无睹 . 但是 , 学生们一开 始 , 就把揭露的主动权掌握在自己手里 . 也许这使他们问心无愧 ,  不过它既不能满足正义的要求 , 也不能满足我本人的自尊心 . 

《学校公报》在今天的一期上否认这些揭露;另一方面 , 根据许 多理由看来 , 我觉得这封信是伽罗瓦写的 . 我认为 , 全校不能再容 忍这个人所犯的严重的过失了 , 从发现罪犯的那一时刻起 , 我们不 能跟他同住在一个屋檐下 . 因此我愿意自己承担责任把他开除出 校 , 迟迟作出我在去年和今年初曾经二十次想做而未做的事情 . 

事实上 , 伽罗瓦是个自从考进本校以来我就不断听到教师和 学监埋怨的唯一的学生 . 但是 , 我因为器重他的不容置疑的数学 天才 , 同时又不相信自己的印象——因为我有个人对他不满的根 据——我容忍他的反复无常的行为 , 容忍他的懒散和不能容人的

性格 , 希望纵使不能改变他的性格 , 哪怕让他有可能读完两年课程 也好 , 这样可使大学不至失去它对这个学生所应行期待的东西 , 同 时又不至使他的母亲痛苦;据我所知 , 她是不得不考虑自己儿子的 未来的 . 我的一切尝试全是徒劳无益 , 反而自讨一场没趣 . 从上 星期日以来 , 我就确信 , 他是无可救药的了 . 也许这个青年早已没 有道德基础 . 

大臣阁下 , 并非我个人因为受到《学校公报》的侮辱才要求采 取措施来结束这份报纸每天在大学的心脏表演危险的把戏 . 但 是 , 既然我领导着第一所高等学府 , 那就该让我对旨在分化师生并 到处传播不信任和纠纷的公开阴谋表示遗憾 . 师范大学用不着害 怕这种卑鄙的攻击 , 因为现有情况鲜明地显示出了在这里学习的 青年们的洁白无瑕的精神状态;学生们举止坚定、谨慎、有分寸;我 替他们负责 , 他们也替我负责 . 但是 , 在我们当中一发生就被清除 出去的祸害正在向着其他学校传播 , 这些学校是不受年龄和教育 抵制的 . 这种悲惨的后果我们已经是有目共睹的了 . 大臣阁下 ,  我快要写完了 . 我所谈到的破坏社会秩序的行为 , 不可能逃过您 的注意 . 毫无疑问 , 不久 , 人们将依样采取旨在恢复纪律的措施; 也就是我们有权期待大学的主要领导人采取的、否则无法进行教 学的措施;就我们而言 , 它是工作必不可少的条件 , 正如秩序是自 由的必不可少的条件一样……

\begin{flushright}
	(全国档案馆F 17卷宗 , 70355)
\end{flushright}

\begin{center}
	\emph{1830年12月12日《宪法报》摘录}
\end{center}

我们认为有责任提请国民教育大臣阁下注意一项滥用职权的 事例 , 师范大学一个最有才华的学生竟成了滥用职权的牺牲品 .  不久以前 , 在一家报纸上可以读到一封署名为该校一学生的信 , 他 在信中批评了该校校长在七月事件期间的行为 . 我们且撇开这封信究竟有多大根据的问题不谈 , 但是我们惊悉:该校校长指责我们所提到的那个学生是写这封信的罪人 , 不经进一步的审查就禁止他来校上课 . 

这位年轻人恳求大臣庇护无效 , 因而被迫辍学 . 我们希望 , 已经证明自己善于在争执的问题中发现真理的美利鲁(Merilhou)先生 , 一定会把这桩案件了解清楚 . 

\section{埃瓦里斯特 \textbullet 伽罗瓦的诉讼案}

\begin{center}
	摘白1831年6月16日《辩论杂志》

塞纳省陪审法庭 . 《布尔根饭店》案件 . 指控教唆 谍害法国国王人身和生命的未遂罪
\end{center}

在星期六发生了不可容忍的破坏秩序的事件以后 , 已经采取 措施来防止外人进入会议厅 . 证人席上和律师席附近几乎全无旁 听者 . 

被告自称:埃瓦里特斯 \textbullet 伽罗瓦 , 二十岁 , 私人讲授数学课 , 生 于布尔-拉-林;辩护律师是窦本 . 

检察官起诉书申诉如下:

今年5月9日在坦普尔郊区布尔根饭店有二百人集会庆祝特雷拉(Trelet)、卡芬雅克等人被宣判无罪 . 晚会在窗朝花园的底层大厅举行 . 与会者不断举杯致词 , 粗鲁地抨击政府;为山93年的革命、为山岳派\footnote{Montagne , 法国资产阶级革命时期最激进的政党 . ——译者}、为罗伯斯庇尔\footnote{罗伯斯庇尔(Robespierre , 1758—1794) , 法国资产阶级革命的杰出活动家 , 雅各实派专政时的革命政府首脑 . ——译者}干杯 . 但为1789年和1830 年的革命举杯祝酒都被拒绝了 . 

一个身着国民自卫军炮兵制服的与会者举杯提议:“为1831年的太阳干杯!愿它象1830年那样炽热 , 但不要那样使人眼花!” 发言人姓名不明 . 每次举杯都夹杂着呼声:“共和国万岁!山岳派 万岁!国民公会万岁!”同时也高喊:“打倒路易-腓利浦!”

正当晚会进行热烈之时 , 埃瓦里斯特 \textbullet 伽罗瓦站了起来——据他自己承认——手持匕首 , 说道:“为路易-腓利浦干杯 . ”他重复举杯说了两次 . 有几个人也模仿他 , 举手喊道:“为路易-腓利浦干 杯!”这时有口哨声 . 这次祝酒的可怕意义引起了一些人愤慨 , 另 有一些人 , 伽罗瓦说 , 以为他真的是在提议为法国国王的健康干 杯 . 然而 , 可以确切断定:许多应邀参加晚会的人对发生的事情大 声表示不赞成 . 

伽罗瓦的匕首是在6月向制刀匠昂利(Henry)定购的 . 他急 于要得到匕首 , 说是他即将出外旅行 . 

法庭庭长南丁(Nandin)开始讯问 . 被告当即承认 , 他确曾出 席那次有两百人参加的集会 . 

问:这次举行集会的动机是什么?

答:因为炮手们被宣判无罪 , 主要的是 , 腊斯拜拒绝荣誉军团 的十字勋章 . 

问:你当时坐在哪里?

答:在大厅中央 , 主席左边 . 

问:你们当时举杯说了些什么?

答:为1793年的革命、为罗伯斯庇尔干杯 . 其余的现在记不 清了 . 

问:谁提议“为1831年7月的太阳”干杯?

答:我不能说 . 我不知道 . 

问:在这以后 , 是不是响起了叫喊声:“快点 , 快点!”

答:是的 , 先生 . 

问:谁喊的?

答:大家都喊 . 

问:是不是还提议为国民议会和山岳派干杯?

答:是的 , 先生 , 但这并不比为1793年的革命和为罗伯斯庇尔 干杯来得次数更多 . 

法庭庭长:你是不是说过这样的话:“为路易-腓利浦干杯” , 同时掏出藏在衣襟下面的匕首?

伽罗瓦:情况确实如此 . 我有一把餐用的刀子 . 我举起这把 刀 , 说:“为路易-腓利浦干杯 , 要是他背叛的话 . ”后一句话只有我 邻座的几个人听见 . 其余的人因为只听清楚前半段话 , 以为我是 在提议为路易-腓利浦的健康干杯 , 就吹起口哨来 . 

问:这样说来 , 你认为在这个集会上为国王陛下的健康干杯是 有点不妥当啰?
答:当然 . 

问:那么 , 简单而明确地提议为法国国王路易-腓利浦干杯就 引起与会者的指责吗?

答:是的 , 先生 . 

问:这就是说 , 你不让路易-腓利浦成为背叛者 , 才用刀威胁 他吗?

答:是的 , 先生 . 

问:你是不是用这种方式表达你个人的这种意见:即法国国王 理应吃上一刀 , 或者 , 你是竭力想鼓动某一个人采取这类行动?

答:只有当路易-腓利浦成为背叛者 , 也就是越出法纪的范围 , 企图加强对人民的剥削时 , 我才想鼓动人家采取这种行动 . 

问:是什么使你认为国王陛下可能违法乱纪?

答:一切都促使我想到有这种可能性 . 

问:请解释一下 , 你这是指什么而言?

答:我想说的是 , 政府的行动不难使人认为 , 有朝一日 , 路易-腓利浦可能背叛 . 

问:那么 , 政府的活动使你认为 , 法国国王陛下有朝一日可能 背叛国家啰?

答:我并不断言 , 路易-腓利浦一定会背叛国家 . 但是我有根 据认为 , 他可能这样做 . 他并没有提供充分的保证 , 能使我们免掉 这种怀疑 . 

问:这就是说 , 你认为在国王的思想和意念中 , 有恶意的企 图吗?

答:是的 , 庭长先生 . 

问:你很懂得向你提出的问题吗?你是在用自己的回答故意 污辱法国国王陛下 . 

伽罗瓦:我简单地说“是的" , 我说得不够清楚 . 我要说的是 , 王的一举一动 , 虽然还没有证明他的不当 , 但使人怀疑他是否正直无私 . 例如 , 他的登极是预先策划的 . 

窦本:为了被告和整个案件的利益 , 我只想指出一点 . 有关被 告个人的感情问题 , 可能把审理引导到既不能使法院、陪审员 , 也 不能使任何人感到愉快的方面去 . 如果问被告 , 政府的哪些命令 和行动使得他怀疑国王陛下的善良意图 , 庭长先生就是迫使辩护 人作出他认为不适宜的解释 . 例如 , 在国王陛下登极之前 , 曾经发 生过一些事件 , 连我自己也能说出这些事件的十分有趣的细节来 .  不过 , 我觉得不必扯得这么远 . 

庭长:律师并不处于被告席上 . 同时 , 我并不要求他讲述他的当事人的意图 . 话又说回来 , 如果对被告的讯问进行得不够仔细 ,  检察官可以对此进行上诉 . 

副检察官米勒尔(Miller):庭长当然应该主持审理 . 不过 , 如 果允许我表示什么愿望的话 , 那我完全同意辩护人的意见:不应该 从这方面进行审理 . 

窦本:这样一来 , 所有的人都会有好处 , 特别是这里所暗示的人 . 

庭长(对被告):从什么时候起你就随身带着这把刀?

伽罗瓦:从5月7日起 . 宴会是在9日举行的 . 这不过是我的癖好 . 法国医生们去照料波兰的伤员时随身带的刀子跟我的刀子相象 . 当时我得不到这种刀 . 一遇到合适的时机 , 我就定购了一把 . 

陪审员杜克罗(Ducros):这种形状的刀子在市场上经常可以 看见 . 去年我自己也有一把刀子象……

庭长:诸位陪审员先生都很清楚 , 陪审员不能以证人身分参与 审理 . 

有人把一把跟被告定购的相同的刀子放在被告面前桌子上 . 

伽罗瓦:这确实是这个诉讼案中的非常重要的细情!共和政体主义者在布尔根饭店集会时 , 就用这种刀子来切母鸡和火鸡 . 

庭长:这么说来 , 其他应邀与会的人也有这种刀子?

伽罗瓦:他们只使用我的 . 

庭长:你的刀子后来怎么啦?

伽罗瓦:那天晚上离开饭店的时候 , 我把它丢了 . 

初级法院执达员协会主任彼提(Petit)以第一证人身分被传出 庭 . 他供称 , 5月9日法院执达员协会在布尔根饭店的一间单独 大厅里举行年宴 . 宴会参加者在花园里散步时 , 从半开半掩的窗户里听见一个二百人的集会发出的叫喊声、祝酒声和歌声 . 

皇家法院律师德勒尔(Delair)申述 , 他曾参加这个二百人的集 会 . 他没有听见有什么人单纯提议为共和国干杯 , 尽管这个词出现在许多人的话里 . 证人看见 , 坐在桌子另一端的被告站了起来 ,  手里拿着亮晃晃的象刀刃一样的东西 . 

他还没有听清举杯致词的内容 , 接着就是一片喧哗 . 

庭长:有没有人向被告建议重说一遍?

德勒尔:我想 , 有人这样建议 . 

庭长:被告举起刀子 , 并说“为路易-腓利浦干杯”之后 , 是不 是还说“如果他背叛的话”?

德勒尔:我没有听见 , 但这完全是有可能的 , 因为经过解释后 叫喊立即停止了 . 

这时门外发生了骚乱 . 有几家报馆的采访员 , 因为服务员不 准他们进入法庭 , 他们就大声要求庭长取消他的决定 . 

庭长准许一些能够证明自己职业的采访员入庭 . 

听众之一:法庭有一半座位还空着 . 

米勒尔:不要以为 , 今天还象前几天那样 , 可以靠近法庭审判 的座位 . 

庭长(对被告):这么说来 , 你承认自己犯煽动罪 , 这种煽动是 具有象征牲的?

伽罗瓦:如果我干脆提议为路易-腓利浦的死亡干杯 , 你们当 然会感到更加高兴了!

庭长:你大大侮辱了这里在场的人 , 尽管你不知道他们的 意图 . 

伽罗瓦:我认为我知道得很清楚 . 

米勒尔:证人在书面供词中曾引用几句祝词 , 那是当着他在场 时说的 . 

德勒尔:我说过 , 有人提议为崇高理想的难忘时代干杯 . 

布尔根饭店餐厅主管德尼(Denis)佩带着有三色丝带的领章 出庭 . 他供称 , 二百名与会的来宾占了一间专用大厅 . 朝花园的 窗户是半开着的 . 

饭店服务员杜兰顿(Durandon)和德泽刻尔(Desesquelle)供 称 , 他们在晚餐结束时听见“共和国”和“革命”这两个字眼 , 但他们无法提供详细情节 . “我忙着收拾餐后的银餐具 , ——德泽刻尔说——请你们相信 , 这事比所有其他工作都使我费劲 . ”

饭店掌柜鲁(Roux)供称:与会的人曾经为1831年的共和国的繁荣干杯 . 

庭长:你所说的主要是指1831年7月吧?当时是不是有人 说:“愿它比1830年7月更加炽热”?

鲁:是的 . 有一个人说:“1831年的革命万岁!”还谈到1830年 的大会 . 

法院执达员库埃(Couet) , 彼伦(Peron) , 克列顿(Creton)供述跟他们的同事彼提所供述的一样的事实 . 

另一位自从7月战斗受伤以来一只手就用跨带吊着的证人供述 , 离他相当远的伽罗瓦举杯致词 , 引起一片喧嚷 . 其中一位穿着炮兵制服的来宾走到伽罗瓦跟前 , 兴奋地和他交谈起来 . 

庭长(对被告):这个身穿炮兵制服的人建议你离开宴会厅吗?

伽罗瓦:这个人(古斯塔夫 \textbullet 德罗伊诺[Gustave Drouineau]) 将以见征人身分出庭 . 他向我走近来 , 只是为了要求我解释我所 说的话 . 

肉商格勒(Guéret)供称:“当我在花园里散步时 , 布尔根饭店 的主人沙尔烈(Charlier)先生向我走来 , 说道:‘老实说 , 参加集会的这伙人我实在不喜欢 . ’——‘可不是 , 我看——我对他说 , —— 他们喧嚷得太厉害 . ’——‘这些人说的话真叫人不寒而栗 . 人家 只好耸耸肩膀 . ’”

庭长:你听到了什么?

格勒:我听见高呼:“共和国万岁!”向一个叫腊斯拜的人祝 贺 , 因为他拒绝接受军团十字勋章 . 不过我一直呆在花园里 , 不想 参加到老爷们里边去 , 他们的行为很不体面 , 甚至在大厅里也吸 烟 . 这等事在这里从来不曾有过 . 

庭长:他们有没有谈到断头台?

恪勒:我听见他们高呼“杀死腓利浦第一!送他的一家人上断 头台!”

伽罗瓦:证人有没有威胁到花园去的来宾?

格勒:我没有威胁过任何人 . 

伽罗瓦:证人威胁过很多人 , 特别威胁过文学家埃仁 \textbullet 普拉尼 奥尔(Eugène Planiol) . 如果法庭同意听他的供述 , 他可以证实这 点 . 他在楼梯下等着 . 

格勒:我没有打扰过这班先生们 , 我跟他们不认识 . 我根本什 么也不知道 . 我只不过说 , 他们对共和国的爱会卡住商业的脖 子……这是我的看法 , 我坚持这种看法 , 因为我跟其余的人一样 ,  吃过苦头 . 

庭长:流露这种情感不会有任何害处 . 

卖刀的昂利夫人供称 , 她要价十四法郎把被告定购的刀子卖 给被告 . 两年前已开始制造这种刀子 , 不过卖得不很多 . 

作家古斯塔夫 \textbullet 德罗伊诺被传出庭作证人 , 他穿着有三色绦 条的黑色燕尾服出庭 . 他的外表引起人们极大的兴趣 . 出席者记 得 , 法院侦察员已因他拒绝作证判处他二百法郎罚款 , 他说明拒绝 作证的理由是:根据法律 , 不能要求提供发生在友好宴会上的事情 的情报 . 

庭长:请举起右手 . 

德罗伊诺:我不想宣誓 , 你从侦查笔录中应当知道 , 我觉得自 己既没有义务 , 也不愿意宣扬发生在私人宴会上的任何事情 . 我 决不想侮辱司法 , 但我认为有权不回答有关本案的任何问题 . 

庭长:我正告你 , 在司法面前 , 每个人都必须就他被问的任何 事实提供证词 , 除非由于自己职业的关系 , 被讯问者才可以不提供 按法律必须保密的情况 . 


德罗伊诺:这次情况特殊 . 我不能供述在节日的友好气氛中 发生的事情 . 

庭长:你坚持你不能提供证词?

德罗伊诺:这是我的正式声明 . 

副检察官:这是你送给法院侦查员的信:

“既然我被传出庭作证 , 我首先要抗议某些报馆大事渲染我在 离开宴会厅之前所说的话 , 我离开那里是因为有人举杯说了一些 与我的政治见解相抵触的话……我拒绝宣誓 , 因为我没有义务 , 也 不想宣扬发生在私人宴会上的任何事情 . 如果我掌握与国家安全 攸关的秘密 , 我就会一分钟也不迟疑地供述出来 , 因为这是我的责 任 . 但在现在情况下 , 凭良心和名誉来说 , 我觉得自己是正确的 .  我从来不曾当过告密人 . 总而言之 , 如果根据民法判我有罪 , 我的 良心会证明我无罪 . ”

米勒尔接着说:现在可以让我说几句话吗?

德罗伊诺:这种让我……说几句也会变成审问的 . 

副检察官:我想只说几句 . 也许这几句能使您信服 , 也许不 能 . 您自己在信里发现一个非常正确的差别 . 您说 , 您无论如何 也不愿意当告密人 , 然而却会毫不迟疑地供述与国家安全攸关的 秘密 . 但是在这种情况下 , 每个公民不仅必须根据司法的要求提 供证词 , 而且必须自己主动地供出他所知道的各种未遂罪和阴谋 . 

德罗伊诺:荣誉的法律不是写在会烂掉的纸头上 , 而是铭刻在 心里的 . 良心告诉我 , 我不应当就一次友情流露的时刻所发生的 事情提供证词 . 

副检察官:很多证人在这个问题上都抱着另一种看法 . 鉴于 德罗伊诺先生拒绝宣誓 , 我们控诉他有罪……

德罗伊诺:不行……我已经被判处罚款 . 我记得有条规则:一 事不重罚两次(non bis in idem) . 

米勒尔:您犯了一次已经判了的罪 , 您现在又犯另一次 . 我们 要求根据刑法第八 . 条判处证人再罚款一百法郎 . 

德罗伊诺:如果我应当受两次处罚 , 我也服从 . 我不打算违反 民法 . 

庭长:您是不是希望法庭建议律师替您辩护?

德罗伊诺:不 . 我不改变我的主意 . 

窦本:我不想就证人上述的供词节外生枝 , 不过我希望指出 ,  既然案情没有宣布 , 根本不应该传唤证人 . 

德罗伊诺:窦本先生 , 谢谢您的好意 , 不过我不打算聘请辩 护人 . 

法官们退席 , 到议事室去 , 经过一小时的商议以后 , 宣布判决 如下:

“按任何证人根据合法理由被传出庭 , 每次都必须提供证词 ,  只要这种证词系属必不可少的;倘证人拒绝履行这项要求 , 每犯一 次 , 理应再受处分;按证人不可就他应传作证的事实作出判断 , 并 且无权以司法上所侦查的事件并非犯罪行为为理由而回避作证; 按所指的证人并不属于法律责成保密的人 , 本庭特判处古斯塔 夫-德罗伊诺先生罚款一百法郎 , 并赔偿诉讼费用 . ”

德罗伊诺(离开法庭):我觉得 , 法庭并未按照我所提到的non bis in idem规则办事 . 

庭长:本庭研究过你的论据 , 才作出判决 . 

开始讯问应被告请求出庭的证人 . 药剂师勒康特(Lecomte) ,  宴会组织者之一 , 还有桂拉尔(Gouillard)、比拉尔(Billard) , 奥杜英(Audouin)、卡马伦(Camalon)和丘佩尔(Cuper)供称 , 当被告举 杯祝词 , 说出前半段话“为路易-腓利浦干杯!”时 , 在与会者当中开始发生大骚动 . 嘈杂声吵得听不见后半段“如果他背叛自己的誓言刀的话 . 当伽罗瓦解释并补充说明自己的意思后 , 口哨声和嘘嘘声就为掌声所代替了 . 

窦本请求庭长运用职权使法庭听取宴会主持人尤贝尔(Hubert)先生和腊斯拜先生的证词; 这次庆祝宴会有一部分也是 为后 者而组织的 . 两位证人正在法庭上 . 他们立即被传作证 . 

尤贝尔 , 年约四十岁、过去当过公证人 , 供称:我们约定不举 杯 , 不说任何非预先通知过宴会的组织者和主席的祝词 . 很多次 举杯致词都是涉及共和国和国民议会的 , 但所有纪念国民议会的 祝词 , 都是为了颂扬它的成员在紧要关头所表现的大无畏精神和 爱国主义 . 有人为罗伯斯庇尔和山岳派举杯祝酒是非常不对的 .  可能在大厅某一个角落里某些与会者私下表示 , 希望有这一类的 祝酒 , 但实际上没有人说出这一类祝词 . 为1831年革命的糟糕 的祝酒是一个令人遗憾的误会 . 我曾经提议为1789年和1830 年的革命干杯 , 为参加第一次革命时期攻克巴士的狱的战斗和第 二次革命时期的巷战而荣膺勋章的公民德孔俾兹(Decombise)干杯 . 公民德孔俾兹想致答词 , 但激动和谦虚妨碍他清楚地表达自 己的思想 . 他请邻座一个人代表他致词 . 邻座那个人想提议为参 加过巷战的年轻战士、为1830年革命的全体参加者干杯 . 他说错了 , 说成1831年 , 不过一发现因激动而说错话 , 他立即为自己 的错误表示歉意 . 还有一次祝词也是完全说错的 . 原话是:“为7 月的太阳干杯!愿它使我们温暖 , 但不要再使人眼花!”有人完全 不正确地把这句话说成是新革命的号召 . 令人遗憾的是 , 类似这 种误解也载在检察长的起诉书里 , 被告本人在某种程度上也承认 了这种误解 , 把一些与会者根本没有的意图硬加到他们的头上 . 

伽罗瓦:责备被告轻率 , 是替他辩护的坏方法 . 

尤贝尔:你把我必须为您本人的利益而予驳倒的某些看法说 成是全体与会者的看法 . 

作家腊斯拜:我坐在宴会的主席右边 , 听见桌子另一端一片喧嚷声 . 刚才这里有人说 , 客人之一高呼:“路易-腓利浦万岁!” 以后又听见口哨声 . 我声明:在我们这次的集会上 , 对路易-腓利 浦既没有人表示支持 , 也没有人表示反对 . 我们的集会是友谊的 ,  而不是政治性的 . 真正的共和主义者基于自己的原则 , 永远不会 为某一个个别人物干杯;因为人们在变 , 而事业却依然如故 . 你们 永远听不到我们的拥护者之中有任何人会喊:“某某人万岁!”因 为一个人今天是我们的朋友 , 明天可能变成我们的敌人 . 后来我才知道 , 伽罗瓦先生是秩序混乱的起因 , 但我既没有听见祝词 , 也 没有听见解释 . 

副检察长米勒尔表示要发表起诉词:被告以敢于发表自己的主张自豪 , 但是今天他第一次企图缓和他所说的那些不能容忍的话 . 直到现在 , 他还说 , 虽然他十分明确地号召人们谋害路易 \textbullet 腓利浦 , 却补充说:“如果他背叛自己的誓言 . ”在侦查期间他的全部回答和这种新情况毫不符合 . 

检察机关在审理公开性的问题时 , 所依据的是服务员的供词 , 从这些供词中可以得出结论 , 透过半敞开的窗户 , 大部分在宴会厅 里所说的话在外边都听得见 . 在上诉法院的无数案例中说明 , 大 饭店和小饭馆基本上都是公共场所 . 因此很明显 , 被告犯有在公 共场所教唆谋害国王陛下的生命和侵犯人身的未遂罪 . 

伽罗瓦当即提出各种各样的解释 , 有一部分是临时作出的 , 另一部分是事先写好的 . 他声称 , 在思想上他是跟这一个月中到陪审法庭受审的全体爱国者站在一起的 , 因此他和他们一样 , 有被判有罪 , 或被判无罪的同样理由 . 

他说:“上星期六我没有坐在这个被告席上 , 我也没有受到拳 头的威胁 , 这都是不由我作主的 . ”

伽罗瓦接下去说:“主张复辟的人 , 欣赏一下你们自己的劳动 果实吧!你们许诺今后不会再有暴动者 . 但暴动者有的是……查理十世要比你们机灵一百倍 . 

庭长:您这不是为自己辩护 , 为您的利益起见……倒不如把您 的话转告自己的律师 . 

伽罗瓦:我快说完了……你们都是小孩!你们把我们的头放 到断头台上 , 可是你们没有力气把斧头砍下去 . 

——我们也是小孩 , 但是我们精力充沛 , 勇往直前 . 共和主义 者的灵魂是污泥沾染不了的 . 而你们 , 你们也是小孩 , 是想竭力 ……如果我们能够说出一切 , 那么那些控告我们的人就会出来干涉 . 请他们不要把我们的沉默当作同意的表示……

庭长:为了您本人的利益起见 , 我现在要打断您的话 . 

伽罗瓦:您一点也不妨碍我 , 我已经说完了 . 

被告辩护人窦本只就公开性的问题加以分析 . ——若干天以前 , 上诉法院撤销道塞弗(Deux-Sèvres)省陪审法院因为德 \textbullet 拉 \textbullet 杜拉 \textbullet 杜 \textbullet 潘 \textbullet 库维涅(de ia Tour du Pin Gouvenet)在酒店发表叛乱言论而判处徒刑的决定 . 撤销该判决系由于缺乏犯罪的必要内容 , 因为陪审法院未曾确定上述言论是当众发表的 . 

窦本先生将这项原则运用于本案 , 从而得出结论 , 认为被告没有任何犯法行为 . 

法庭庭长南丁就讨论作了总结 . 

经过半小时的商讨之后 , 陪审员们对向他们提出来的问题作 了答复:

“不 , 被告无罪 . ”

埃瓦里斯特-伽罗瓦被公认无罪并获得释放 . 

\section{埃瓦里斯特 \textbullet 伽罗瓦的数学著作}

1 . 蒙彼利埃\footnote{蒙彼利埃(Montpellier) , 法国南部城市名 . ——译者}科学系教授J . -D . 热尔贡(Gergonne)出版的定期文集《纯粹与实用数学年鉴》内载:

埃瓦里斯特 \textbullet 伽罗瓦 , 路易-勒-格兰中学学生 , 证实一项循环连分数定理 . 第19卷 , 1828 — 1829年 , 第294页 . 

埃瓦里斯特 \textbullet 伽罗瓦 , 师范大学学生 , 分析方面若干问题的笔记 , 第 21 卷 . 1830—1831 年 . 

2 . 费律萨克(Férussac)出版的《数学、物理和化学通报》内载:
埃 \textbullet 伽罗瓦 , 一份有关解代数方程的研究报告的分析 , 1830年 4月笔记本 , 第271页 . 

埃 \textbullet 伽罗瓦 , 有关解代数方程的笔记 , 有关数论的笔记 , 1830年6月笔记本 , 第413页和428页 . 

死后发表的有:

上述作品 , 以及

致奥古斯特 \textbullet 舍瓦烈的信;

关于根式可解方程之条件的研究报告;

根式可解的方程 . 发现在伽罗瓦的文件中 , 1846年在约瑟 夫 \textbullet 刘维的《纯粹与实用数学杂志》(第10卷)上发表 . 

埃 \textbullet 伽罗瓦数学作品集1897年由法国数学协会出版 , 埃米 尔 \textbullet 皮卡尔作序(哥底叶-威腊尔父子出版社) . 1951年又以新版 本出版 . 

未收入作品集的伽罗瓦的个别数学笔记 , 由朱利 \textbullet 汤内里发表在《数学通报》(第2集第30-31卷 , 1906—1907年版)上 . 这些笔记于1908年以单行本形式出版 , 定名为《埃瓦里斯特 \textbullet 伽罗瓦 手稿》(哥底叶-威腊尔父子出版社版) . 
